
\begin{frame}[fragile]
\frametitle{Motivation}

Libraries:
\begin{itemize}
\item collections of data and behavior
\item have well-defined interfaces 
\item suitable to be used by multiple independent programs.
\end{itemize}
${}$\\
  Library usage dramatically increase programmers' productivity.
\\${}$\\
Today a well-designed program can use up to 10 third-party libraries. However, when more than 10 libraries are used keeping trace of the issues arising from those libraries and their interplay become very difficult.
\end{frame}

\begin{frame}[fragile]
\frametitle{Today, a goal for tomorrow}

Today a programmer can use a library containing \alert{millions of instructions} without the need to understand each instruction individually, and instead can rely on the library interface.
\\${}$\\
 With 42, a programmer will be able to use 
 \alert{millions of libraries} without the need to understand each library individually, and may rely on libraries managing and organizing other libraries.
\\${}$\\
This would alter the perception of what programming is: 
Normally the programming language is considered 
``the thing the programmers use to code''.
The philosophy of 42 is different: the language  provides the foundation for libraries, and libraries is what programmers use to code.
 
\end{frame}

\begin{frame}[fragile]
\frametitle{Smaller libraries}

Today libraries tends to be massive chunks of code with
tons of functionalities packed into.
\\${}$\\
This is because importing and managing libraries is hard,
so is convenient for libraries to "expand" their domain
in related areas.
\\${}$\\
With easy - transparent - libraries,
There will be no reason to have so many functionalities packed
all together.
\\${}$\\

\end{frame}


\begin{frame}[fragile]
\frametitle{Purpose}
In 42 a \emph{library} is a self-contained piece of code that is released
by other programmers.
A good library has a modularized documentation and 
contains a significant amount of code dedicated to try and
deliver good error messages when used incorrectly with links to the relevant documentation section.
\\${}_{}$\\
In 42 an \emph{active library} is a library offering methods that generate other libraries.
Such methods can compose code (other libraries) to obtain highly personalized libraries that can provide a clear dedicated semantic to the programmer.
\end{frame}

\begin{frame}[fragile]
\frametitle{Purpose: Primary features}
\begin{itemize}
\item  using (active) libraries is simple;

\item  developing/deploy (active) libraries is simple;

\item  using good (active) libraries produces good error messages;

\item  the language semantics/typing  helps to isolate erroneous (and even bad) libraries;

\end{itemize}
\begin{Scaled}{0.8}{0.8}
\parbox{\linewidth}{
Non-goals: trying to accomplish this goals would harm our core goals:
size of the binary,
memory footprint of the objects and performance predictability.}
\end{Scaled}
\end{frame}

%\fullPageImage{marvin1}

%\begin{frame}[fragile]
%\frametitle{Overview of the talk}
%\begin{itemize}
%\item Language features and syntax
%\item Getting started with Active Libraries
%\item Type system of 42
%\end{itemize}
%\end{frame}

%\fullPageImage{typing1}

%\begin{frame}[fragile]
%\frametitle{Minimal OO capabilities}

%\begin{itemize}
%\item (final static inner) classes with state (constructor composed by fields) and behaviour;
%\item single dispatch method call, with nominal method selectors;
%\item interfaces as the only way to induce subtyping; that is, all class types are exact types.
%\end{itemize}
%\begin{NiceCode}{59ex}{0.70}
%\begin{lstlisting}
%{
%A: {()}$\Comment{A is an empty class}$
%B: {(A fieldName)
%  C: {(A f1 B f2)
%    method C foo(C c1 C c2) {
%      return C(f1:c1.f1(), f2:c2.f2())
%      $\Comment{a new C using parameters as fields}$
%      }   }
%  method C (C c) {
%    return C.foo(c1:c, c2:c)  $\Comment{method selector are composed by}$
%    $\Comment{method name+parameter names. Method name can be empty.}$
%    $\Comment{see how C(f1, f2) used in foo is a normal method call}$
%    }    }  }
%\end{lstlisting}
%\end{NiceCode}
%Since there is no ``extends'' code reuse is obtained with active libraries.
%\end{frame}


\begin{frame}[fragile]
\frametitle{Decorators}

Library composition is obtained using decorators, that are build upon 
basic decorators \Q@Use@ and \Q@Adapt@.\\*
${}_{}$\\*
Composing classes ``as modules'',\\*
\pause with a associative and commutative use/merge/sum
operation:\\*
\begin{itemize}
\pause\item nested classes with same name are recursively merged;\\*
\pause\item the set of implemented interfaces is merged;\\*
\pause\item methods with same name are merged,\pause and conflict arises if both version have an implementation or if the headers are not compatible.\\*
\end{itemize}
${}_{}$\\*
\pause Good, but... how to merge fields/constructors?

\end{frame}


%where should this go?
\begin{frame}[fragile]
\frametitle{Constructors and fields}

%  type method Library point()Export"Point"<{
%    Point:Data[abstract:#]<{(N x, N y)
%      
%      }
%    }

\pause What is exactly a field/constructor in 42,
${}_{}$\\*
\pause since I can only
access them as getters/setters/factories?
${}_{}$\\*
\pause What is the exact meaning of writing
\Q@Point:{(N x, N y)}@?
${}_{}$\\*
\pause
It is just sugar for 
\begin{NiceCode}{59ex}{1}
\begin{lstlisting}
Point:{ $\Comment{(N x, N y) desugar into}$
  type method Outer0 (N^ x, N^ y)
  method N x()
  method N y()
  }
\end{lstlisting}
\end{NiceCode}

That is, a class with just abstract methods.
\end{frame}


\begin{frame}[fragile]
\begin{NiceCode}{59ex}{1}
\begin{lstlisting}
Point:{ $\Comment{(N x, N y) desugar into}$
  type method Outer0 (N^ x, N^ y)
  method N x()
  method N y()
  }
\end{lstlisting}
\end{NiceCode}
\pause In many languages, a concrete class have no abstract methods;
In 42, a concrete class have a \alert{coherent set} of abstract methods:
\begin{itemize}
\pause\item A class with no abstract methods is concrete {\small(as {\texttt{java.lang.Math}})}
\pause\item A class with a single abstract \Q@type@ method returning \Q@Outer0@
whose parameters are all of placeholder type (as \Q@N^ x@) is concrete.
\pause\item If such abstract \Q@type@ method is present, then, for each parameter,
it is allowed to have abstract getters and setters,
where the name maps back to the parameter name.
\end{itemize}
\pause
Take a deep breath, to me this is game changing and mind blowing.
There are only methods. The constructor/getter implementation is not
protected/hidden/secret/native. It is just \alert{not there}.
\pause The whole formal model works without requiring any difference between abstract methods
and getters/setters/constructors.

\end{frame}

\begin{frame}[fragile]
\frametitle{Composing Constructors and fields}


\begin{NiceCode}{59ex}{0.9}
\begin{lstlisting}
C:{
  type method Library point(){(N x, N y)
    method N distance(Outer0 that){
      x=this.x()-that.x()
      y=this.y()-that.y()
      return (x*x+y*y).sqrt()
      }

    }
  }
ColPoint:Use[C.point()]<{(N x, N y, Color color)
  type method Outer0 (N^ x,N^ y)
    Outer0(x:x,y:y,color:Color.black())
  }
\end{lstlisting}
\end{NiceCode}
\end{frame}

\begin{frame}[fragile]
\frametitle{Composing Constructors and fields}
\addtocounter{framenumber}{-1}

\begin{NiceCode}{59ex}{0.9}
\begin{lstlisting}
C:{
  type method Library point(){(N x, N y)
    method N distance(Outer0 that){
      x=this.x()-that.x()
      y=this.y()-that.y()
      return (x*x+y*y).sqrt()
      }
    method Outer0 clone(N x) Outer0(x:x,y:this.y())
    }
  }
ColPoint:Use[C.point()]<{(N x, N y, Color color)
  type method Outer0 (N^ x,N^ y)
    Outer0(x:x,y:y,color:Color.black())
  }$\Comment{now with sets the color to black :-(}$
\end{lstlisting}
\end{NiceCode}
\end{frame}

\begin{frame}[fragile]
\frametitle{Composing Constructors and fields}
\addtocounter{framenumber}{-1}

\begin{NiceCode}{59ex}{0.9}
\begin{lstlisting}
C:{
  type method Library point(){(N x, N y)
    method N distance(Outer0 that){
      x=this.x()-that.x()
      y=this.y()-that.y()
      return (x*x+y*y).sqrt()
      }
    method Outer0 clone(N x)
    }
  }
ColPoint:Data[]<<Use[C.point()]<{(N x, N y, Color color)
  type method Outer0 (N^ x,N^ y)
    Outer0(x:x,y:y,color:Color.black())
  }$\Comment{let data do its job}$
\end{lstlisting}
\end{NiceCode}
\end{frame}


\begin{frame}[fragile]
\frametitle{Traits, trivially}
Traits are just groups of methods
\\*
(and they can implement interfaces)
\\*
\Q@{<:Foo,Bar ..method T x().. method Void foo().. }@
\\*
Just add traits together to get other threads.
\\*
\Q@Use[Lib.a();Lib.b();..]@
\\*
Sum a trait to a (concrete) class and get
as a result a better (concrete) class, if such class offers all
the methods required (i.e. abstract) in the thread.
\\*
\Q@Use[Lib.a();Lib.b();..]<{..}@
\\*


\end{frame}


\begin{frame}[fragile]
\frametitle{Expression problem as matrix}

\renewcommand{\arraystretch}{2}
\newcommand\open{
\begin{NiceCode}{30ex}{0.85}
}

\newcommand\close{
\end{NiceCode}
}

$$ 
\begin{Scaled}{0.75}{0.75}
\begin{tabular}[t]{ c | c | c |}
              & \texttt{toString} &\texttt{eval}\\\cline{1-3}

\texttt{Num}\!\!\!&
\open\begin{lstlisting}
{
Exp:{interface<:HasToString }
Num:{(N that)<: Exp
  method toString()
    S""++this.that()
  }}
\end{lstlisting}\close

%numString
%\LocalTableCell{\expToStringNumA}{8ex}
&
\open\begin{lstlisting}
{
Exp:{interface<:HasEval }
Num:{(N that)<: Exp
  method eval()
    this.that()
  }}
\end{lstlisting}\close

%numEval
%\LocalTableCell{\expEvalNumA}{8ex}
\\\cline{1-3}
\texttt{Sum}\!\!\!&
\open\begin{lstlisting}
{
Exp:{interface<:HasToString }
Sum:{(Exp left, Expr right)<: Exp
  method toString()
    this.left().toString()
    ++this.right().toString()
  }}
\end{lstlisting}\close

%sumString
%\LocalTableCell{\expToStringSumA}{11ex}
&
\open\begin{lstlisting}
{
Exp:{interface<:HasEval }
Sum:{(Exp left, Expr right)<: Exp
  method eval()
    this.left().eval()
    ++this.right().eval()
  }}
\end{lstlisting}\close

%sumEval
%\LocalTableCell{\expEvalSumA}{11ex}
\\\cline{1-3}
\texttt{UMinus}\!\!\!&
\open\begin{lstlisting}
{
Exp:{interface<:HasToString }
UMinus:{(Exp that)<: Exp
  method toString()
    S"-"++this.that().toString()
  }}
\end{lstlisting}\close

%umString
%\LocalTableCell{\expToStringUMinusA}{8ex}
&
\open\begin{lstlisting}
{
Exp:{interface<:HasEval }
UMinus:{(Exp that)<: Exp
  method eval()
    this.that().eval()*-1N
  }}
\end{lstlisting}\close

%umEval
%\LocalTableCell{\expEvalUMinusA}{8ex}
 \\\cline{1-3}
\end{tabular}
\end{Scaled}
$$

\renewcommand{\arraystretch}{1}
\end{frame}


\begin{frame}[fragile]
\frametitle{Expression problem as matrix}
Just sum the desired line/rows together
\begin{NiceCode}{59ex}{0.8}
\begin{lstlisting}
MySolution:ExprProblem(
  useRows:SList[S"uMinus";S"sum"]
  useColums:SList[S"eval";S"toString"]
  )
\end{lstlisting}
\end{NiceCode}




\end{frame}



\begin{frame}[fragile]
\frametitle{Adapt}
Ok, \Q@Use@ is cool.
What about \Q@Adapt@?
It can rename methods, nested classes, and can set members private.
More than that, \Q@Adapt@ is cool since it can apply \Q@Use@ internally.
\begin{NiceCode}{59ex}{0.8}
\begin{lstlisting}
D:Adapt[Name"C" into: Name"K";Name"A" into: Name"K"]<{
  C:{method N c()}
  A:{method N a()}
  method C m(A a)
  }
\end{lstlisting}
\end{NiceCode}
rewrites to 
\begin{NiceCode}{59ex}{0.8}
\begin{lstlisting}
D:{
  K:{method N c() method N a()}
  method K m(K a)
  }
\end{lstlisting}
\end{NiceCode}
\end{frame}


\begin{frame}[fragile]
\begin{NiceCode}{59ex}{0.8}
\frametitle{Adapt}
\begin{lstlisting}
D:Adapt[ Name"C::A" into Name"Outer0" ] < {
  method N top()
  C:{method N c()
    A:{ method N a() }
    }
  }
\end{lstlisting}
\end{NiceCode}
rewrites to 
\begin{NiceCode}{59ex}{0.8}
\begin{lstlisting}
D:{
  method N top()
  method N a()
  C:{method N c() }
  }
\end{lstlisting}
\end{NiceCode}
Should this be called \Q@Move@ instead of \Q@Adapt@?

\end{frame}


\begin{frame}[fragile]
\frametitle{Adapt}
\begin{NiceCode}{59ex}{0.8}
\begin{lstlisting}
D:Adapt[ Name"Outer0" into Name"C::A" ] < {..}
\end{lstlisting}
\end{NiceCode}
rewrites to 
\begin{NiceCode}{59ex}{0.8}
\begin{lstlisting}
D:{ C:{ A:{..}}}
$\Comment{or D:\{interface C:\{interface A:\{..\}\}\}}$
\end{lstlisting}
\end{NiceCode}
Should this be called \Q@Move@ instead of \Q@Adapt@?
And finally,

\begin{NiceCode}{59ex}{0.8}
\begin{lstlisting}
D:Adapt[ Name"C" into Name(S) ] < {
  C:{.. B:{..}}
  method C m(C::B x)
  }
\end{lstlisting}
\end{NiceCode}
rewrites to 
\begin{NiceCode}{59ex}{0.8}
\begin{lstlisting}
D:{
  method S m(S::B x)
  }
\end{lstlisting}
\end{NiceCode}
This last operation requires to check the structural type of
\Q@S@ and \Q@S::B@ to be compatible with \Q@C@ and \Q@C::B@.
Should this be called \Q@Replace@ instead of \Q@Adapt@?
\end{frame}



\begin{frame}[fragile]
\frametitle{Towel Intro}
A towel is about the most massively useful thing a programmer can have.
A towel has immense psychological value, and you should always know where your towel is.
\begin{NiceCode}{59ex}{1}
\begin{lstlisting}
{reuse $\ReuseUrl{L42.is/BasicTowel}$
Main:{
  Debug(S"Hello world")
  return ExitCode.normal()
  }
}
\end{lstlisting}
\end{NiceCode}
Here \Q@Debug@, \Q@S@ and \Q@ExitCode@ are classes defined inside the towel
BasicTowel.
You can code without a towel, but this means starting from first principles,
witch could be quite unpleasant; especially since the only
primitive things that 42 offers are Library literals
(code as first class entities), the constant \Q@void@,
and the types \Q@Library@, \Q@Void@ and \Q@Any@.
\end{frame}

\begin{frame}[fragile]
\frametitle{Towel Intro}
The header reuse L42.is/BasicTowel imports the code from 
the URL L42.is/BasicTowel and places it at the start of the class.
Code is(=will be) downloaded/updated from the Internet automatically.
We generally call towels libraries that are expected to provide standard
functionalities and types, such as numbers N, booleans Bool,
strings S and various kinds of system errors
(and Load, as discussed later).
However, we do not expect all L42 programs to reuse the same towel.
For hygienic reasons, in real life everyone tends to use their own towel.
For similar reasons, any 42 program will use its own towel.
We expect different programs to use massively different libraries for
what in other languages is the standard library.
That is, there is no such thing as ''the L42 standard library''
\end{frame}
\begin{frame}[fragile]
\frametitle{Towel Load}


Most L42 libraries are not towels.
All 42 libraries are closed code.
\\*
Thus, all non-towel 42 library is generic
(have abstract classes/methods)

\begin{NiceCode}{59ex}{1}
\begin{lstlisting}
{reuse $\ReuseUrl{L42.is/BasicTowel}$
Gui:Load[]<{reuse $\ReuseUrl{L42.is/2dGui}$}
Main:{
  Gui.alert(S"Hello world")
  return ExitCode.normal()
}
\end{lstlisting}
\end{NiceCode}
\Q@Load[]<{..}@
would intuitively behave like 
\Q@Adapt[Name"N" into:Name(N);Name"S" into:Name(S);...]<{..}@\\*
\Q@Load@ uses \Q@Adapt@ to redirect such abstract classes to their incarnation in BasicTowel,
assuming they respect some structural interface.

\end{frame}
\begin{frame}[fragile]
\frametitle{Towel Multiple Towels}

Towels shines when multiple towels are used at the same time.

\begin{NiceCode}{59ex}{1}
\begin{lstlisting}
{reuse $\ReuseUrl{L42.is/AdamsTowel}$
$\Comment{ here you can access to lots of}$
$\Comment{ utility classes declared inside the towel}$
$\Comment{ including numbers, strings and so on.}$
C:{reuse $\ReuseUrl{L42.is/FordTowel}$
  $\Comment{here you can access a different set of classes.}$
  $\Comment{For example, N would refer to the number in FordTowel}$
  $\Comment{and to see the number declared in AdamTowel}$
  $\Comment{you have to write Outer1::N}$
  }
}
\end{lstlisting}
\end{NiceCode}

Different code parts reason about different set of classes;
including those predefined in other languages.

Useful for code that reasons on code; that is a very common task
in L42. 


\end{frame}
\begin{frame}[fragile]
\frametitle{Towel Multiple Towels}

Define and deploy our own towel: like BasicTowel, with Gui as before.
\begin{NiceCode}{59ex}{1}
\begin{lstlisting}
{reuse $\ReuseUrl{L42.is/DeploymentTowel}$
Main:Deploy[URL".../GuiTowel";permissions:S"..."]<{
  reuse $\ReuseUrl{L42.is/BasicTowel}$
  Gui:Load[]<{reuse $\ReuseUrl{L42.is/2dGui}$}
  }
}
\end{lstlisting}
\end{NiceCode}
DeploymentTowel is just some variation of BasicTowel, with Deployment
functionalities already included.
Here \Q@Load@ comes from BasicTowel, and the nested code literal
results in closed code, thus \Q@Deploy@ will be able to correctly
export it and save it in the provided URL, if we have permission.

\end{frame}
\begin{frame}[fragile]
\frametitle{Towel Embroidery}
We can mix and match different implementations for numbers and strings (as an example).
This requires some special care:
\begin{itemize}
\item strings should know their length (as a number)
\item numbers should be comparable (into true/false booleans)
\item booleans (and numbers) should be representable (as strings)
\end{itemize}
\begin{NiceCode}{59ex}{0.9}
\begin{lstlisting}
{reuse $\ReuseUrl{L42.is/DeploymentTowel}$
Main:
  Deploy[URL".../MyTowel";permissions:S"..."]<<
  Adapt[Name"NAlt" into:Name"N"; Name"SAlt" into:Name"S"]<<
  Abstract[Name"N";Name"S"]<$\Comment{now N and S are abstract}$
    {$\Comment{code with alternative versions for N and S}$
    reuse $\ReuseUrl{L42.is/BasicTowel}$
    NAlt:Load[]<{reuse $\ReuseUrl{L42.is/InfinitePrecisionNatural}$}
    SAlt:Load[]<{reuse $\ReuseUrl{L42.is/UTF8Strings}$}
    }
\end{lstlisting}
\end{NiceCode}

\end{frame}
\begin{frame}[fragile]
\frametitle{Library Deployment}

Non-towel libraries can also be deployed in a similar way:
\begin{NiceCode}{59ex}{0.9}
\begin{lstlisting}
{reuse $\ReuseUrl{L42.is/DeploymentTowel}$
Main:
  Deploy[URL"../MyTowel";permissions:S".."]<<
  Optimize::RemoveUnreachableCode[]<<
  ExposeAsLibrary[Name"MyLib"]<{
    reuse  $\ReuseUrl{L42.is/BasicTowel}$
    MyLib:... $\Comment{this refers to the file MyLib.L42}$
    } }
\end{lstlisting}
\end{NiceCode}


Where \Q@ExposeAsLibrary[Name"MyLib"]<{..}@
could be:
\begin{NiceCode}{59ex}{0.9}
\begin{lstlisting}
  Adapt[makePrivate:Name"PrImpl"]<<
  Adapt[Name"PrImpl::N" into:Name"N";Name"PrImpl::S" into:..]<<
  Adapt[Name"PrImpl::MyLib" into:Name"Outer0"]<<
  Adapt[Name"Outer0" into:PrImpl]<<
  Abstract[Name"N";Name"S";Name"Bool";..]<{..}
\end{lstlisting}
\end{NiceCode}

\end{frame}




\begin{frame}[fragile]
%\addtocounter{framenumber}{-1}
\frametitle{Support for decent size applications}

\begin{NiceCode}{59ex}{0.8}
\begin{lstlisting}
{reuse $\ReuseUrl{L42.is/AnotherDeploymentTowel}$
Local:UrlData.readOnly(prefix:URL"file:///../myProjects/")
Remote:UrlData.writeOnly(prefix:URL"L42.is/byMarco/",permission:S"..")
Web:UrlData.writeOnly(prefix:URL"www.myWebsite.com/",permission:S"..")
Header:Deployer[]<{()
    method DeployResult() {
      var acc=Project(Local"header")
      b= Benchmark(Local"headerBenchmark")
      acc :=b.optimize(acc, inlining:10Minute)
      acc :=b.optimize(acc, parallel:1Minute)
      return acc.deploy(Remote"header")
      }
  }
GraphicEngine:Deployer[]<{(Header h)
  method DeployResult() {
    var acc=Project(Local"graphicEngine")
      $\Comment{inside reuse L42.is/byMarco/header}$
    b1=Benchmark(Local"graphicEngineBenchmark1")
    b2=Benchmark(Local"graphicEngineBenchmark2")
    acc =b1.optimize(acc, inlining:10Minute)
    acc =b2.optimize(acc, parallel:10Minute)
    return acc.deploy(Remote"graphicEngine")
   }
 }
..
\end{lstlisting}
\end{NiceCode}
\end{frame}


\begin{frame}[fragile]
%\addtocounter{framenumber}{-1}
\frametitle{Support for decent size applications}
\begin{NiceCode}{59ex}{0.8}
\begin{lstlisting}
MyGame:Deployer[]<{(Header h GraphicEngine e)
  method DeployResult() {
    var acc=Project(Local"myGame")
    b=Benchmark(Local"myGameBenchmark")
    acc =b.optimize(acc, inlining:10Minute)
    acc =b.optimize(acc, parallel:10Minute)
    return acc.deploylist()[
        windowsExe: acc Web"myGame.exe";
        linuxExe: acc   Web"myGame";
        javaExeJar: acc Web"myName.jar";
        htmlPage: Local"index.html".asText())
          name: Web"index.html";
      
      ]
    }
  }
Main:Build(MyGame)$\Comment{Build compute dependency graph and check for changes}$
\end{lstlisting}
\end{NiceCode}
\end{frame}






%
%\begin{frame}[fragile]
%\frametitle{Getting started}
%\begin{NiceCode}{59ex}{1}
%\begin{lstlisting}
%reuse $\ReuseUrl{L42.is/base}$                                       
%Main:{
%  Gui::Alert(S"hello world")
%  Gui::Alert(S"hello world",  title:S"Dialog Title")
%  Gui::Alert(S"today is "+Date.today().asText())
%  return ExitCode.normal()
%  }
%\end{lstlisting}
%\end{NiceCode}
%
%Here, \Q@reuse@ is importing code from a pre-existing library.
%No need for library installation.
%
%Note: the only primitive types are \Q@Any@, \Q@Void@ and \Q@Library@, thus any other class is implemented \emph{in} the language, as a library.
%%This includes strings \Q@S@, numbers \Q@N@ and the class decorator \Q@Data@.
%\end{frame}

%\fullPageImage{hand1}

%\begin{frame}[fragile]
%\frametitle{Getting started with Active Libraries}
%How an Active Library looks like? 
%\\*${}_{}$
%\\*
%Minimal Active Library
%
%\begin{NiceCode}{59ex}{1}
%\begin{lstlisting}
%A:{() $\Comment{a class with a method m returning the empty class}$
%  method Library m() {return {()}}
%  }
%B: A().m() $\Comment{an empty class generated using library A}$
%\end{lstlisting}
%\end{NiceCode}
%rewrites to 
%
%\begin{NiceCode}{59ex}{1}
%\begin{lstlisting}
%A:{() $\Comment{a class with a method m returning the empty class}$
%  method Library m() {return {()}}
%  }
%B:{()} $\Comment{the empty class}$
%\end{lstlisting}
%\end{NiceCode}
%
%\end{frame}
%
%\begin{frame}[fragile]
%\frametitle{Incremental compilation}
%{\small Like compile time execution in C++, template haskell and others.
%First class code literals, called \Q@Librar@ies allows classes
% to be created, from top to bottom, using formerly defined classes.
%In the end, the execution process itself is the production of all the classes.}
%
%\begin{NiceCode}{59ex}{0.8}
%\begin{lstlisting}
%$\Comment{(STEP0)}$
%A:{()$\Comment{a class with a method k producing an A}$
%  method Library m() {$\Comment{m just returns a class with a method k}$
%    return {() method A k(){return A()}} $\Comment{ returning an A instance}$
%  }}
%B:A().m()
%C:B().k().m()
%\end{lstlisting}
%\end{NiceCode}
%
%$\begin{array}{l | l}
%\mbox{During execution becomes} & \mbox{and finally}\\
%\!\!\begin{NiceCode}{25ex}{0.8}
%\begin{lstlisting}
%$\Comment{(STEP1)}$
%A: ... $\Comment{as before}$
%B:{() method A k(){return A()}}
%C:B().k().m()
%\end{lstlisting}
%\end{NiceCode}
%&
%
%\begin{NiceCode}{25ex}{0.8}
%\begin{lstlisting}
%$\Comment{(STEP2)}$
%A: ... $\Comment{as before}$
%B:{() method A k(){return A()}}
%C:{() method A k(){return A()}}
%\end{lstlisting}
%\end{NiceCode}
%\end{array}$
%
%{\small
%that is the result of the execution of the original 
%program.}
%\end{frame}
%

\begin{frame}[fragile]
\frametitle{Getting started with Active Libraries}
\begin{NiceCode}{59ex}{1}
\begin{lstlisting}
reuse $\ReuseUrl{L42.is/AdamsTowel}$                                       
AgeInput: Gui.input(receive:N, request:S"insert age")
$\Comment{class AgeInput is the first class we declare!}$
$\Comment{is generated by method "Gui.input"}$
Main:{
  Gui::Alert(S"your age is "+AgeInput().asText())
  $\Comment{class AgeInput principal method (the one with no name)}$
  $\Comment{ask the user to insert a number N}$
  return ExitCode.normal()
  }
\end{lstlisting}
\end{NiceCode}
\end{frame}


\begin{frame}[fragile]
\frametitle{Getting started with Active Libraries}
\begin{NiceCode}{59ex}{1}
\begin{lstlisting}
reuse  $\ReuseUrl{L42.is/base}$
BirthInput: Gui.input(
  title:S"Date of birth"
  receive:Date $\Comment{we ask a widgets for the type Date}$
  request:S"insert your date of birth")
Main:{
  Date userDate= BirthInput()
  Gui::Alert(S"your date of birth is "+userDate.asText())
  return ExitCode.normal()
  }
\end{lstlisting}
\end{NiceCode}
\Q@Gui.input@ can generate widgets for each type.
\end{frame}



\begin{frame}[fragile]
\frametitle{Getting started with Active Libraries}
\begin{NiceCode}{59ex}{1}
\begin{lstlisting}
reuse  $\ReuseUrl{L42.is/base}$
Tables: DB(connection:DBConnection"....")
  $\Comment{create a class with a nested for each db table}$
  $\Comment{plus utility features to connect with the db}$
...
me=Tables::Student(id:....)
  $\Comment{get student from id, queries also available.}$
....me.name()....
\end{lstlisting}
\end{NiceCode}

\Q@DB@ turns a db into classes.
If the db is updated, and \texttt{name} is
renamed in \texttt{fullName},
42 will point to the error place and we will be guided into
correspondingly updating our program.

\begin{NiceCode}{59ex}{1}
\begin{lstlisting}
reuse  $\ReuseUrl{L42.is/base}$
Tables: DB(connection:....)$\Comment{ with the updated db-schema}$
...
me=Tables::Student(id:....)
....me.name()....$\Comment{now I get a red sign here!}$
\end{lstlisting}
\end{NiceCode}
\end{frame}

%\begin{frame}[fragile]
%\frametitle{Getting started}
%\begin{NiceCode}{59ex}{1}
%\begin{lstlisting}
%reuse  $\ReuseUrl{L42.is/base}$
%BirthInput: Gui.input(
%  title:S"Date of birth"
%  receive:Date $\Comment{input produce widgets for any type}$
%  request:S"insert your date of birth")
%Date userDate= BirthInput()
%$\pause$Second userS= Date.today() - userDate
%years= (userS/Seconds.year().asNumber()).floor()
%Gui::Alert(S"you are "+years.asText()+S" years old")
%myBirth=Date.gregorian(
%  year:1981N
%  month:Month::April
%  day:21N)
%if myBirth<userDate (
%  Gui::Alert(S"
%    $\Comment{I'm younger that you}$
%    $\Comment{       Ha ha ha!!}$
%    "))$\Comment{ here an example of multi line string}$
%\end{lstlisting}
%\end{NiceCode}
%\end{frame}

\begin{frame}[fragile]
\frametitle{Visualize DB}
\begin{NiceCode}{59ex}{1}
\begin{lstlisting}
{reuse  $\ReuseUrl{L42.is/AdamsTowel}$
Db: Load[]<{reuse L42.is/Db}
Gui: Load[]<{reuse L42.is/2dGui}
Tables: DB(S"...my connection string...")
StudentList: Collections.list(Tables::Student)
StudentsView: Gui.tableOf(StudentList)
QueryFrom: Tables::Query[result:StudentList](
  DB::SQL"Select * from Student where country=@country")
Main:{
  connection=Tables.connect()
  fromItaly=QueryFrom(connection, country:S"Italy")
  Gui.show(StudentsView(fromItaly))
  return ExitCode.normal()
  }
}
\end{lstlisting}
\end{NiceCode}
\end{frame}



\begin{frame}[fragile]
\frametitle{Visualize DB}
\begin{NiceCode}{59ex}{1}
\begin{lstlisting}
DB: Load[]<{$\ReuseUrl{L42.is/DB}$}
Gui: Load[]<{$\ReuseUrl{L42.is/2dGUI}$}
Tables: DB(S"...my connection string...")
StudentList: Collections.list(Tables::Student)
StudentsView: Gui.tableOf(StudentList)
QueryFrom: Tables::Query[result:StudentList](
  DB::SQL"Select * from Student where country=@country")
\end{lstlisting}
\end{NiceCode}
\begin{itemize}
\item \Q@Load[]<{$\ReuseUrl{L42.is/..}$}@
\Q@Load@ is loading the code of a library
and, according to the conventions in 
$\ReuseUrl{L42.is/AdamsTowel}$, is linking numbers, strings
and other general purposes classes to their correspondent 
inside of $\ReuseUrl{L42.is/AdamsTowel}$.
\item \Q@DB(S"...my connection string...")@
creates classes to reify tables in the database.
\item \Q@StudentList: Collections.list(Tables::Student)@
creates a collection for students.
\end{itemize}
\end{frame}


\begin{frame}[fragile]
\frametitle{Visualize DB}
\begin{NiceCode}{59ex}{1}
\begin{lstlisting}
StudentsView: Gui.tableOf(StudentList)
QueryFrom: Tables::Query[result:StudentList](
  DB::SQL"Select * from Student where country=@country")
\end{lstlisting}
\end{NiceCode}
\begin{itemize}
\item \Q@StudentsView: Gui.tableOf(StudentList)@
create a widget visualizing lists of students.
Every collection can be seen as a table, if the object supports
iteration over its fields and its class supports iteration over its field names and types.
\Q@Data[]@ can easily add those methods for you.
\item \Q@QueryFrom: Tables::Query[result:StudentList](..)@
creates a (class that can only perform a) prepared query.
If the query is of the form \Q@"Select * from TableName where ..."@ then
it can produce a list of instances of a class under \Q@Table::@.
Otherwise, it would also generate a nested class for the result.
\end{itemize}
\end{frame}

\begin{frame}[fragile]
\frametitle{Visualize DB}
\begin{NiceCode}{59ex}{1}
\begin{lstlisting}
Main:{
  connection=Tables.connect()
  fromItaly=QueryFrom(connection, country:S"Italy")
  Gui.show(StudentsView(fromItaly))
  return ExitCode.normal()
  }
\end{lstlisting}
\end{NiceCode}
\begin{itemize}
\item \Q@Tables.connect()@ connects to the database.
\item\Q@fromItaly=QueryFrom(connection, country:S"Italy")@
Using \Q@QueryForm@ we ask for students coming from Italy.
Notice how \Q@country:@ is part of the method signature.
The name of the parameter was extracted while generating the class.
A good IDE autocompletion would take advantage of it.
\item\Q@Gui.show(StudentsView(fromItaly))@
close the circle, showing a table view of our Italian students.

\end{itemize}
\end{frame}


\begin{frame}[fragile]
\frametitle{Maintainability?}
\begin{NiceCode}{59ex}{1}
\begin{lstlisting}
{reuse  $\ReuseUrl{L42.is/AdamsTowel}$
Db: Load[]<{reuse L42.is/Db}
Gui: Load[]<{reuse L42.is/2dGui}
Tables: DB(S"...my connection string...")
StudentList: Collections.list(Tables::Student)$\Comment{ HERE!!}$
StudentsView: Gui.tableOf(StudentList)
QueryFrom: Tables::Query[result:StudentList](
  DB::SQL"Select * from Student where country=@country")
Main:{
  connection=Tables.connect()
  fromItaly=QueryFrom(connection, country:S"Italy")
  Gui.show(StudentsView(fromItaly))
  return ExitCode.normal()
  }
}
\end{lstlisting}
\end{NiceCode}
\end{frame}
