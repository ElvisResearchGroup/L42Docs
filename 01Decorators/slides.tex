
\begin{frame}[fragile]
\frametitle{First exposure to 42}
\begin{NiceCode}{59ex}{0.8}
\begin{lstlisting}
Point: {(N x, N y) $\Comment{class point and field declarations}$
  method N distance(Point that){$\Comment{method declaration}$
    var tmpX=this.x()-that.x() $\Comment{fields are getters}$
    tmpX:=tmpX*tmpX $\Comment{allowed since declared var}$
    var tmpY=this.y()-that.y() $\Comment{omitted var type}$
    tmpY*=tmpY  $\Comment{usual Java/C++ operators}$
    return (tmpX+tmpY).sqrt() $\Comment{pure oo view}$
    }
  method Point  add(N x){$\Comment{method declaration}$
    return Point(x:this.x()+x, y:this.y())$\Comment{constructor from fields}$
    }
  }
\end{lstlisting}
\end{NiceCode}

Minimal OO capabilities:
\begin{itemize}
\item (final) classes with state (constructor composed by fields) and behaviour;
\item single dispatch method call, with nominal method selectors;
\item interfaces as the only way to induce subtyping; that is, all class types are exact types.
\end{itemize}

\end{frame}

\begin{frame}[fragile]
\frametitle{First exposure to 42}
\begin{NiceCode}{59ex}{0.8}
\begin{lstlisting}
Point: { (N x, N y) ... }
Main:{
  p1= Point(x:12N,y:22N) $\Comment{this is declaring a final local var}$
  p2= Point(x:12N,y:22N)
  $\Comment{check=p1==p2 'equivalent of Java .equals}$
  $\Comment{method == is undefined}$
  }
\end{lstlisting}
\end{NiceCode}

With \Q@Point@ as defined before, we can do very little with points.
We may want to manually add
\begin{itemize}
\item equality, inequality, hashCode
\item since all the fields can be compared  (by using \Q@>=@) then
we may want to implement a method comparing points lexicographically
\item toString, serialize, deserialize/parse, toHTML, clone,...
\end{itemize}
Implementing such methods is not only very boring,
\PresentationOnly\pause\alert{It is super tricky!}

\end{frame}

\begin{frame}[fragile]
\frametitle{How can it be so hard?}
\begin{itemize}
\item It is actually order of magnitude  simpler to use objects defining those methods than to define them correctly.
\PresentationOnly\pause\item Up to the point that every year multiple articles challenge  the establish standard on how to
do those kinds of operation on arbitrarily shaped objects/object graphs.
\PresentationOnly\pause\item If automatic generations of those methods was a language feature
%/part of a standard library,
it would be outdated very fast.
%\PresentationOnly\pause\item If it was a third party library, it would be doomed to stay unknown to most programmers. (How many of us known and use Project Lombock?)
%\PresentationOnly\pause\item 42 is designed to attempt solving this kinds of problems.
\end{itemize}
\end{frame}


\begin{frame}[fragile]
\frametitle{First exposure to class Decorators}
\begin{NiceCode}{59ex}{0.8}
\begin{lstlisting}
Point: Data[]<<{ (N x, N y) ... }
Main:{
  p1= Point(x:12N,y:22N) $\Comment{this is declaring a final local var}$
  p2= Point(x:12N,y:22N)
  check=p1==p2 $\Comment{true}$
  less=p1>p2 $\Comment{false}$
  }
\end{lstlisting}
\end{NiceCode}
\Q@Data@  is a class, whose instances are class decorators.
A class decorator is an object offering  a method named \Q@<<@ that takes a library (=a class with nested classes) and return a library; that is:\\*
\Q@method Library <<(Library that)@.
Since class decorator tends to have many options, they can be instantiated using the square brackets, as in \Q@Data[...optionName1:value1;optionName2:value2;...]@  .\\*
(That is the syntax for any var-arg method, including convenient collections initialization as \Q@NVec[a;b;c;]@).

\end{frame}


\begin{frame}[fragile]
\frametitle{More examples of Data}
\begin{NiceCode}{59ex}{0.8}
\begin{lstlisting}
Margin: {(N maxX, N maxY, N minX, N minY)}$\Comment{just the fields}$
\end{lstlisting}
\end{NiceCode}
\PresentationOnly\pause
\begin{NiceCode}{59ex}{0.8}
\begin{lstlisting}
Margin: Data[]<<{(N minX, N minY, N maxX, N maxY)}
$\Comment{automatically generated equals/hashcode, no validation}$
\end{lstlisting}
\end{NiceCode}
\PresentationOnly\pause{\small If \Q@invariant@ is provided, it will be dynamically called in the right moments.}
\begin{NiceCode}{59ex}{0.8}
\begin{lstlisting}
Margin: Data[]<<{(N minX, N minY, N maxX, N maxY)
  method Void invariant() {
    return Assert[
      this.minX() >0N;
      this.minY() >0N message:S"y should be positive";
      this.maxX() >this.minX();
      this.maxY() >this.minY();  ]}}
\end{lstlisting}
\end{NiceCode}
\PresentationOnly\pause {\small Now we can have points inside a margin.}
\begin{NiceCode}{59ex}{0.8}
\begin{lstlisting}
Point: Data[]<<{(Margin that,  N x, N y)
  method Void invariant() {
    return Assert[   that.minX() <=this.x();  that.minY() <=this.y();
       that.maxX() >=this.x();  that.maxY() >=this.y();  ]}}
\end{lstlisting}
\end{NiceCode}

\end{frame}



\begin{frame}[fragile]
\frametitle{Active libraries and Decorators: What is new here?}
\begin{itemize}
\item An active library is a library that have some way to generate code.
\item A decorator is a special kind of active library, that takes a class and ``improve it".

\item If you have some former knowledge of generative programming, you will know that I'm not proposing anything new or groundbreaking.

\PresentationOnly\pause\item What I'm proposing is to design a language around the idea that decorators are going to \alert{be there} and massively used.

\PresentationOnly\pause\item In the end, the programmers may change their perception about writing code: Instead of directly encoding the behaviour they need, they will give some suggestion to a decorator, that is going to do all the hard job.

\PresentationOnly\pause\item This also allows for a much simpler language,requiring only a simple nominal (sub-)types and no inheritance.
\\*Code reuse (with inheritance as a special case), generics (as shown later), enumerations and many other concepts may be encoded as (active) libraries / decorators.
\end{itemize}
\end{frame}


\begin{frame}[fragile]
\frametitle{Getting started with Active Libraries}
How an Active Library looks like? 
\\*${}_{}$
\\*
Minimal Active Library

\begin{NiceCode}{59ex}{1}
\begin{lstlisting}
A:{() $\Comment{a class with a method m returning the empty class}$
  method Library m() {return {()}}
  }
B: A().m() $\Comment{an empty class generated using library A}$
\end{lstlisting}
\end{NiceCode}
rewrites to 

\begin{NiceCode}{59ex}{1}
\begin{lstlisting}
A:{() $\Comment{a class with a method m returning the empty class}$
  method Library m() {return {()}}
  }
B:{()} $\Comment{the empty class}$
\end{lstlisting}
\end{NiceCode}

\end{frame}

\begin{frame}[fragile]
\frametitle{Incremental compilation}
{\small Like compile time execution in C++, template haskell and others.
First class code literals, called \Q@Librar@ies allows classes
 to be created, from top to bottom, using formerly defined classes.
The program can perform side effects while creating classes.
Execution is just the production of all the classes.}

\begin{NiceCode}{59ex}{0.8}
\begin{lstlisting}
$\Comment{(STEP0)}$
A:{()$\Comment{a class with a method k producing an A}$
  method Library m() {$\Comment{m just returns a class with a method k}$
    return {() method A k(){return A()}} $\Comment{ returning an A instance}$
  }}
B:A().m()
C:B().k().m()
\end{lstlisting}
\end{NiceCode}

$\begin{array}{l | l}
\mbox{During execution becomes} & \mbox{and finally}\\
\!\!\begin{NiceCode}{26ex}{0.8}
\begin{lstlisting}
$\Comment{(STEP1)}$
A:{() ..} $\Comment{as before}$
B:{() method A k(){return A()}}
C:B().k().m()
\end{lstlisting}
\end{NiceCode}
&

\begin{NiceCode}{26ex}{0.8}
\begin{lstlisting}
$\Comment{(STEP2)}$
A:{() ..} $\Comment{as before}$
B:{() method A k(){return A()}}
C:{() method A k(){return A()}}
\end{lstlisting}
\end{NiceCode}
\end{array}$

{\small
that is the result of the execution of the original 
program.}
\end{frame}



\begin{frame}[fragile]
\frametitle{Minimal building block}
To manipulate libraries we do not use any templating mechanism.
Libraries can be composed with other libraries using an algebra of composition operators /decorators.
The main one is \Q@Compose@, that decorates a class by summing to it a list of libraries. It is the only one that uses multiple libraries.\\*
\Q@Compose[a;b;c]<<d@ composes \Q@a b c d@. Abstract methods can be implemented, nested classes with the same name are recursively summed.

\PresentationOnly\pause On default the composition is symmetric, but \Q@Compose@ can take options to automatically solve conflicts (i.e. same method implemented in multiple classes). For example, a right preferential strategy.

Note that generated libraries are decorated copies of old ones.
\\*
Pre-existing code is never modified.
\end{frame}

\begin{frame}[fragile]
\frametitle{Minimal building block}

In addition to compose there are a plethora of \alert{refactoring} operators, that takes a single library and adapt it somehow.
\begin{itemize}
\PresentationOnly\pause\item \Q@RenamePath@ and \Q@RenameSelector@ rename nested classes and methods
\PresentationOnly\pause\item \Q@RemoveImplementationPath@  make a nested class (and the whole subtree) fully abstract. (See also \Q@RemoveImplementationSelector@)
\PresentationOnly\pause\item \Q@Redirect@ delete a  nested class and redirect all the reference to it to an external one.
\PresentationOnly\pause\item \Q@Introspect@ is a class allowing to query a library on its methods and nested classes names and structural shapes.
\PresentationOnly\pause\item \Q@MakePrivatePath@ and \Q@MakePrivateSelector@ to tune privateness.
\PresentationOnly\pause\item \Q@AddDocumentationPath@, \Q@AddDocumentationSelector@, \Q@RemoveDocumentationPath@, \Q@RemoveDocumentationSelector@
support generation of documentation with metaprogramming.
\end{itemize}
\end{frame}


\begin{frame}[fragile]
\frametitle{Collection generation / not with generics}
\begin{NiceCode}{59ex}{0.8}
\begin{lstlisting}
NVector: Collection.vector(N)

PointSet:Collection.set(Point)

StudentList:Collection.orderedList(Student, orderedBy:{
  method Boolean(Student a, Student b) {return a.name>b.name}
  })
\end{lstlisting}
\end{NiceCode}
Collections are generated on demand one for each different type.
Multiple incompatible collections can be generated for the same type, in order to underline a different role for different collections.
\end{frame}

\begin{frame}[fragile]
\frametitle{Collection generation / not with generics}
Thanks to \Q@Refactor::Redirect@ we can emulate generics, just redirect a class to the final destination.
Internally, the code of \Q@Collection.vector@ may look like the following
\begin{NiceCode}{59ex}{0.8}
\begin{lstlisting}
Collection:{...
  Library vector(type Any that){
    return Refactor::Redirect[Path"T" to: that ]<<{()
      T:{}
      method Void add(T that){...}
     ...}
    }
  }
\end{lstlisting}
\end{NiceCode}
And the result of \Q@NVec:Collection.vector(N)@ could be:
\begin{NiceCode}{59ex}{0.8}
\begin{lstlisting}
$\Comment{NVec:Collection.vector(N)}$
NVec:{()
  method Void add(N that){...}
  ...}
\end{lstlisting}
\end{NiceCode}


Note how in this way we only need a simple nominal type system.
\end{frame}


\begin{frame}[fragile]
\frametitle{Pseudo higher order functions}
In the same way, it would be possible to have maps/filters/folds as in functional languages, without the need of polymorphic types
\begin{NiceCode}{59ex}{0.8}
\begin{lstlisting}
myStudents= PersonList[....]
FindOlder: Accumulate[PersonList]<<{
  method Person accumulate (Person left, Person right){
    if left.age()>right.age() ( return left )
    return right
  }}
older= FindOlder.from(myStudents)
$\Comment{haskell-like equivalent}$
$\Comment{older=accumulate (left,right->if left.age()>right.age() left else right) myStudents}$
\end{lstlisting}
\end{NiceCode}
While do not look as nice as an haskell-like equivalent,
our approach works with a simple pure nominal type system.
\\*
\PresentationOnly\pause
\alert{Question:}
\\*
Is it easier to reason about decorators as producing code taking some hint in input, or
is it easier to reason about higher order polymorphic function composition?
\end{frame}


\begin{frame}[fragile]
\frametitle{Compose}
 \Q@Compose@ decorates a class by summing to it a list of libraries.
Since there is no ``extends'', code reuse is obtained with decorators.
For example, to define a gui with callbacks, we can have a html form where buttons have ids, and we can implement the onclick method:

%Test this code in the prototype!
%should be ok with Gui g and give type error with Outer1 g
\begin{NiceCode}{59ex}{0.8}
\begin{lstlisting}
Gui:Compose[GuiFromHtml(myHtml)]<<{
  ButtonStart:{ method Void onClick(Gui g){
    g.resultText().value(g.inputText().value())   }}
\end{lstlisting}
\end{NiceCode}

\Q@GuiFromHtml@ have a set of conventions: generate nested classes for elements with associated events, and getters for elements with an \Q@id@ attribute.\\*
The programmer can/must rely on those conventions!\\*
As an alternative design, \Q@GuiFromHtml@ could be a decorator; this it could examine the input code and check conventions are respected.
\begin{NiceCode}{59ex}{0.8}
\begin{lstlisting}
Gui:GuiFromHtml[myHtml]<<{
  ButtStart:{ method Void onClick(Gui g){$\Comment{error is discovered}$
    g.resultText().value(g.inputText().value())   }}
\end{lstlisting}
\end{NiceCode}
\end{frame}


\begin{frame}[fragile]
\frametitle{Resolver}
The need of super is satisfied with resolvers. In the following code
\begin{NiceCode}{59ex}{0.8}
\begin{lstlisting}
Compose[
  { method Bool a(){return True()} };
  { method Bool a(){return False()} };
  resolver:{
    Bool a(){return this.left() & this.right()}
    Bool left()
    Bool right()  
    };
  ]
\end{lstlisting}
\end{NiceCode}
The \Q@Compose@ instance will compose the two classes successfully and
the resulting method \Q@a()@ will perform the logic and of the result of the two methods.

\end{frame}


\begin{frame}[fragile]
\frametitle{How Data could work? - Example: equals on fields}
Example of input and expected result
\begin{NiceCode}{59ex}{0.76}
\begin{lstlisting}
Point:Data[]<<{(N x,N y)}$\Comment{example just on ==}$
\end{lstlisting}
\end{NiceCode}
reduces to
\begin{NiceCode}{59ex}{0.76}
\begin{lstlisting}
Point:{(N x,N y)
  method Bool == (Any that){
    with that (
      on Outer0  return this.internEquals(that) 
      default    return false 
      )
    }

  method $\Comment{@private}$
  Bool internEquals(Outer0 that){
    return this.x() ==that.x() & this.y() ==that.y()
    }
  }
\end{lstlisting}
\end{NiceCode}

\end{frame}


\begin{frame}[fragile]
\frametitle{How Data could work? - Example: equals on fields}
\begin{NiceCode}{59ex}{0.74}
\begin{lstlisting}
method Library equal(type Any class,Selector name) {$\Comment{equal on single field}$
  return Refactor::Redirect[Path"T" to: class]<<
         Refactor::RenameSelector[Selector"f()" to:name]<<{
    T:{ method Bool ==(Any that)}
    method T f()
    method Bool internEquals(Outer0 that) {return this.f()==that.f()}
    }  
  }$\PresentationOnly\pause$
method Library addEquals(Library l){$\Comment{produces a class with == and internEquals}$ $\PresentationOnly\pause$
  var composer=Compose[resolver:{$\Comment{ how to merge two equal on single field}$
    method Bool internEquals(Outer0 that) {
      return this.left(that) & this.right(that) }
    method Bool left(Outer0 that)
    method Bool right(Outer0 that)   }]$\PresentationOnly\pause$
  with field in Introspection.fieldsOf(l).vals() ($\Comment{accumulate equalities}$
    composer:=composer.add(this.equal(class:field.class(),name:field.name())))$\!\!\PresentationOnly\pause$
  result=composer<<{$\Comment{compute internEquals and add the == method}$
    method Bool internEquals(Outer0 that)
    method Bool ==(Any that){
      with that (
        on Outer0  return this.internEquals(that)
        default    return false
        ) }   }$\PresentationOnly\pause$
  return Refactor::MakePrivateSelector[Selector"internEquals(that)"]<<result}
\end{lstlisting}
\end{NiceCode}
\end{frame}


\begin{frame}[fragile]
\frametitle{Just a different way to generate code}
Operations are defined by inductively generated methods:
\begin{itemize}
\PresentationOnly\pause\item define a behaviour for base case (``single field'' in the example)
\PresentationOnly\pause\item define composition of behaviours (``logic and'' in the example)
\PresentationOnly\pause\item refine the result depending of what you are doing (``result'' in the example) 
\end{itemize}
\PresentationOnly\pause At any step is possible to check what we are creating and provide good error messages if something is going wrong.\\*${}_{}$\\*
Any operation can be synthesized as a library that would implement the operation on the original input when summed (Compose).

\end{frame}


%\begin{frame}[fragile]
%\frametitle{Just a different way to generate code}
%More Decorators! 
%\begin{itemize}
%\item \Q@Abstract[]<<{...}@ produces the skeleton of the input (that is, all the methods become abstract)
%\item \Q@Load[]<<{...}@ takes a library with abstract methods/classes and bind them to some predefined value.
%\item \Q@Iterable[]<<{...}@ using \Q@v.size()@, \Q@v(i)@ and \Q@v(i,val:e)@ adds a standard implementation of iterators for the input.
%\item \Q@Optimize[]<<{...}@ produces a library with the same behaviour but with better performance (this would need ast level reasoning)
%
%\end{itemize}
%
%\end{frame}


%\begin{frame}[fragile]
%\frametitle{Decorator composition}
%Decorators are 
% endofunctions (functions whose domain is equal to its codomain), thus we can apply many decorators in a chain.
%\begin{NiceCode}{59ex}{0.8}
%\begin{lstlisting}
%Point1: Data[]<<Logging[]<<{...}
%\end{lstlisting}
%\end{NiceCode}
%\Q@Logging@ here will somehow log all the methods call of the original code, but not the ones added by \Q@Data@ itself.\\*
%(Note that this works since \Q@<<@, \Q@>>@, \Q@++@ and \Q@**@ are right associative in 42.)
%\end{frame}


\begin{frame}[fragile]
\frametitle{Decorators instead of Quote/Unquote DSL}
What is a Quote/Unquote DSL (also known as a Templating DSL)?
Main way to do metaprogramming.
Example, Java Mint
\begin{NiceCode}{59ex}{0.8}
\begin{lstlisting}
Code<String> msg=..;
Code<Void> myCode=<|
{
  if(Foo.bar){ Foo.doBla();}
  else { `(msg) }
}
|>;
\end{lstlisting}
\end{NiceCode}
Special parenthesis \Q@<|  |>@ denotes first class code literals,
with holes, denoted by the symbol \Q@`( )@.
The holes can be filled with arbitrary expressions of \Q@Code@ type.
This is just a layer of syntactic sugar over regular 
Meta Object Protocol(MOP), that is basically an object oriented data-structure representing
an AST for the language under consideration.
Depending on the specific language,
it is possible to guarantee a range of properties on the resulting code.
\end{frame}

\begin{frame}[fragile]
\frametitle{Decorators instead of Quote/Unquote DSL}

Templating DSL: a low level mechanism for metaprogramming.\\*
Focus is placed on the code and on the concrete implementation.\\*
Since every kind of code can have holes, there is an enormous error space.\\*
${}_{}$\\*\PresentationOnly\pause
In my opinion Templating DSLs are a fast and dirty \\*
way to obtain good initial results using metaprogramming.\\*
The technique does not scale and is very low level.
${}_{}$\\*\PresentationOnly\pause
L42 does \alert{not} rely on Templating DSL.\\*
We focus on behaviour composition instead of ``source code'' composition,
and we take inspiration from class/module composition languages.
\end{frame}


%\fullPageImage{lego1}

\begin{frame}[fragile]
\frametitle{Today, a goal for tomorrow}
The shift of mind that I'm suggesting here is similar to the one from imperative to object oriented programming.
\begin{itemize}
\item With OO, thanks to dynamic dispatch, the programmer do not chose explicitly what behaviour to invoke when a method is called.\\*
(as in \Q@a.draw(b)@)\\*
However, the programmers are still in control, since they are instantiating and passing around such objects.
\PresentationOnly\pause\item With class decorators, the programmers do not chose explicitly what methods are in each class.\\*
(as in \Q@Person:Data[]<<...@)\\*
However, the programmers are still in control, since they writing and applying such decorator objects.
\end{itemize}\PresentationOnly\pause
The difference is in the which programmer keeps the control:
In both cases we have a shift: the programmers of the final application can delegate more and more control to \emph{the programmers of the libraries}.
\end{frame}

%
%\begin{frame}[fragile]
%\frametitle{Today, a goal for tomorrow}
%
%Today a programmer can use a library containing \alert{millions of instructions} without the need to understand each instruction individually, and instead can rely on the library interface.
%\\${}$\\
% With 42, a programmer will be able to use 
% \alert{millions of libraries} without the need to understand each library individually, and may rely on (active) libraries managing and organizing other libraries.
% 
%\end{frame}
%
%
%\begin{frame}[fragile]
%\frametitle{Purpose}
%In 42 a \emph{library} is a self-contained piece of code that is released
%by other programmers.
%A good library has a modularized documentation and 
%contains a significant amount of code dedicated to try and
%deliver good error messages when used incorrectly with links to the relevant documentation section.
%\\${}_{}$\\
%In 42 an \emph{active library} is a library offering methods that generate other libraries.
%Such methods can compose code (other libraries) to obtain highly personalized libraries that can provide a clear dedicated semantic to the programmer.
%\\${}_{}$\\
%This is the way I propose to scale library usage to millions:
%Libraries can be composed by code and not by programmers.
%\end{frame}


\begin{frame}[fragile]
\frametitle{Can you guess what this code is doing?}
\begin{NiceCode}{59ex}{1}
\begin{lstlisting}
{reuse  $\ReuseUrl{L42.is/AdamsTowel}$
DB: Load[]<<{reuse $\ReuseUrl{L42.is/DB}$}
Gui: Load[]<<{reuse f$\ReuseUrl{L42.is/2dGui}$}
Tables: DB(S"...my connection string...")
StudentList: Collections.list(Tables::Student)
StudentsView: Gui.tableOf(StudentList)
QueryFrom: Tables::Query[result:StudentList](
  DB::SQL"Select * from Student where country=@country")
Main:{
  connection=Tables.connect()
  fromItaly=QueryFrom(connection, country:S"Italy")
  Gui.show(StudentsView(fromItaly))
  return ExitCode.normal()
  }
}
\end{lstlisting}
\end{NiceCode}
\end{frame}



\begin{frame}[fragile]
\frametitle{ Thanks! }
\center{\huge Questions?}
\end{frame}

