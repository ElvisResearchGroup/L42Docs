


\begin{frame}[fragile]
\frametitle{Flattening}
\begin{NiceCode}{59ex}{1}
\begin{lstlisting}
A:{ 'a class with ``m'' returning a class with ``foo''
  type method Library m() { N foo() 42N }
  'in Java this would look like 
  'static Library m()\{return \{ int foo()\{return 42;\}\};\}
  }
B: A.m() 'a class generated using library A
\end{lstlisting}
\end{NiceCode}
rewrites to 

\begin{NiceCode}{59ex}{1}
\begin{lstlisting}
A:{ 'a class with ``m'' returning a class with ``foo''
  type method Library m() { N foo() 42N }
  }
B:{ N foo() 42N }'a class 
\end{lstlisting}
\end{NiceCode}
\end{frame}

\newcommand\mini{\begin{NiceCode}{12.55ex}{0.73}}
\newcommand\G{\color{green}{\blacksquare}}
\newcommand\R{\color{red}{\blacksquare}}
\newcommand\Y{\color{darkYellow}{\blacksquare}}
\newcommand\GG{\color{black}{\blacksquare}}
\definecolor{darkYellow}{rgb}{0.94, 0.88, 0.19}
\begin{frame}[fragile]
\frametitle{Flattening}
\mini
\begin{lstlisting}
$\R$N:{..}'number
$\R$S:{..}'string
$\R$A:{.. }'uses N,S
$\Y$B: A.m()'uses A
$\R$D:{..E..}
$\R$C: B.m({..D..})
$\R$E: ..
\end{lstlisting}
\end{NiceCode}
\mini
\begin{lstlisting}
$\G$N:{..}'number
$\G$S:{..}'string
$\G$A:{.. }'uses N,S
$\Y$B: A.m()' A,N,S
$\R$D:{..E..}
$\R$C: B.m({..D..})
$\R$E: ..
\end{lstlisting}
\end{NiceCode}
\mini
\begin{lstlisting}
$\G$N:{..}'number
$\G$S:{..}'string
$\G$A:{.. }'uses N,S
$\G$B: A.m()' ok
$\R$D:{..E..}
$\R$C: B.m({..D..})
$\R$E: ..
\end{lstlisting}
\end{NiceCode}
\mini
\begin{lstlisting}
$\G$N:{..}'number
$\G$S:{..}'string
$\G$A:{.. }'uses N,S
$\G$B:{..}'uses A,N,S
$\R$D:{..E..}
$\R$C: B.m({..D..})
$\R$E: ..
\end{lstlisting}
\end{NiceCode}
\small
\begin{itemize}
\item[$\Y$]
find the first class that requires meta execution.
\item[$\G$]
type-check all the (transitively) used classes, starting from the $\Y$ expression.
\item[$\G$]
type-check the yellow expression. We have enough type information now.
\item[$\G$]
execute the well typed expression. Now we get a class.
By construction, this class can only refer to the types collected before.
Does it means it is already guaranteed to be well-typed?
\end{itemize}
Type-check steps can end the execution with a type error,
while the execution step can end the execution with an exception.

\end{frame}




\begin{frame}[fragile]
\mini
\begin{lstlisting}
$\G$N:{..}'number
$\G$S:{..}'string
$\G$A:{.. }'uses N,S
$\G$B:{..}'uses A,N,S
$\R$D:{..E..}
$\Y$C: B.m({..D..})
$\R$E: ..
\end{lstlisting}
\end{NiceCode}
\mini
\begin{lstlisting}
$\G$N:{..}'number
$\G$S:{..}'string
$\G$A:{.. }'uses N,S
$\G$B:{..}'uses A,N,S
$\GG$D:{..E..}'uses E
$\Y$C: B.m({..D..})
$\R$E: ..'hello?
\end{lstlisting}
\end{NiceCode}
\mini
\begin{lstlisting}
$\G$N:{..}'number
$\G$S:{..}'string
$\G$A:{.. }'uses N,S
$\G$B:{..}'uses A,N,S
$\GG$D:{..E..}'uses E
$\G$C: B.m({..D..})
$\R$E: ..'hello?
\end{lstlisting}
\end{NiceCode}
\mini
\begin{lstlisting}
$\G$N:{..}'number
$\G$S:{..}'string
$\G$A:{.. }'uses N,S
$\G$B:{..}'uses A,N,S
$\R$D:{..E..}
$\R$C:{..}'uses?
$\R$E: ..
\end{lstlisting}
\end{NiceCode}

%new line
\small
\begin{itemize}
\item[$\Y$]
find the first class that requires meta execution.
\item[$\G$]
type-check the used classes.
This time, we have a class literal that uses \Q@D@ (that uses \Q@E@).
What happens here?
\item[$\G$]
type-check the yellow expression. We have enough type information now.
\end{itemize}
Can we proceed without verifying \Q@{..D..}@?
How the decorators/composition operators are going to work over 
a non verified \Q@{..D..}@?


\end{frame}




\begin{frame}[fragile]

%new line

\mini
\begin{lstlisting}
$\G$N:{..}'number
$\G$S:{..}'string
$\G$A:{.. }'uses N,S
$\G$B:{..}'uses A,N,S
$\R$D:{..E..}
$\R$C:{..}'uses E
$\R$E: ..
\end{lstlisting}
\end{NiceCode}
\mini
\begin{lstlisting}
  =====>

  =====>

  =====>

  =====>
\end{lstlisting}
\end{NiceCode}
\mini
\begin{lstlisting}
$\G$N:{..}'number
$\G$S:{..}'string
$\G$A:{.. }'uses N,S
$\G$B:{..}'uses A,N,S
$\R$D:{..E..}
$\R$C:{..}'uses E
$\R$E:{..}
\end{lstlisting}
\end{NiceCode}
\mini
\begin{lstlisting}
$\G$N:{..}'number
$\G$S:{..}'string
$\G$A:{.. }'uses N,S
$\G$B:{..}'uses A,N,S
$\G$D:{..E..}
$\G$C:{..}'uses E
$\G$E: ..
\end{lstlisting}
\end{NiceCode}
\small
\begin{itemize}
\item[$\R$]
Eventually, we resolve also E.
\item[$\R$]
Finally, some class may still be not type-checked
\item[$\G$]
As a last step, we verify all those classes.
\end{itemize}


\end{frame}




\begin{frame}[fragile]
\frametitle{Flattening}
\begin{NiceCode}{59ex}{1}
\begin{lstlisting}
I: {interface
  method N m()   }
C:{  
  type method Library myReusableCode() {<: I
    method m() 42N
    }
  }
D:C.myReusableCode()
\end{lstlisting}
\end{NiceCode}
rewrites to 

\begin{NiceCode}{59ex}{1}
\begin{lstlisting}
I: {interface
  method N m()   }
C: ...
D:{<: I 'note, there is no relation between D and C
    method m() 42N
    }
\end{lstlisting}
\end{NiceCode}
\end{frame}

\begin{frame}[fragile]
\begin{NiceCode}{59ex}{0.8}
\begin{lstlisting}
C:{  
  type method Library a() {
    method N m1(N that) that+1N
    }
  type method Library b() {
    method N m1(N that)   'this is an abstract method
    method N m2(N that) this.m1(that)+1N
    }         ' "this" is of type Outer0, no relation with C
  }
D:Compose[ C.a() ] << C.b()' the result (if any) will be well typed
\end{lstlisting}
\end{NiceCode}
rewrites to 

\begin{NiceCode}{59ex}{0.8}
\begin{lstlisting}
C:{
  type method Library a() {
    method N m1(N that) { return that+1N; }
    }
  type method Library b() {
    method N m1(N that)  
    method N m2(N that) this.m1(that)+1N
    }         ' "this" is of type Outer0, no relation with C
  }
D:{'this code is well typed by construction
  method N m1(N that) that+1N
  method N m2(N that) this.m1(that)+1N
  }         ' "this" is of type Outer0  *and* Outer1::D
\end{lstlisting}
\end{NiceCode}
\end{frame}



\begin{frame}[fragile]
\begin{NiceCode}{59ex}{0.8}
\begin{lstlisting}
C:{    'this code is required to be completely well typed 
         'before D can be generated
  type method Library a() {
    method N m1(N that) that+1N
    }
  }
D:Compose[ C.a() ] << {       'the result (if any) will be well typed
  method N m1(N that)  ' if this code is correct
  method N m2(N that) this.m1(that)+1N
  }         ' "this" is of type Outer0, is it also  Outer1::D
\end{lstlisting}
\end{NiceCode}
rewrites to 

\begin{NiceCode}{59ex}{0.8}
\begin{lstlisting}
C:{  
  type method Library a() {
    method N m1(N that) that+1N
    }
  }
D:{
  method N m1(N that) that+1N
  method N m2(N that) this.m1(that)+1N
  }         ' "this" is of type Outer0  *and* Outer1::D
\end{lstlisting}
\end{NiceCode}
\end{frame}


\begin{frame}[fragile]
Sometimes, we need to delay type checking for code in class literals.
\begin{NiceCode}{59ex}{0.8}
\begin{lstlisting}
A:Data[] << {'the result well-typedness depends from the
            ' code in the dots
  method B playWithB(B that) .. that.fooB() .. 
  ..
  }         
B:Data[] << {'the result well-typedness depends from the
            ' code in the dots
  method A playWithA(A that) .. that.fooA() ..  
  ..
  }         
\end{lstlisting}
\end{NiceCode}
\end{frame}

\begin{frame}[fragile]
By using \Q@D@ I can refer to the type "after" the composition
\begin{NiceCode}{59ex}{0.8}
\begin{lstlisting}
C:{    'this code is required to be completely well typed 
         'before D can be generated
  type method Library a() {
    method N m1(N that) that+1N
    }
  }
D:Compose[ C.a() ] << {'code not required to be checked at this stage.
  method N m2(N that, D other) other.m1(that)+1N
  } 'note, no declaration for m1!
\end{lstlisting}
\end{NiceCode}
rewrites to 

\begin{NiceCode}{59ex}{0.8}
\begin{lstlisting}
C:{
  type method Library a() {
    method N m1(N that) that+1N
    }
  }
D:{'the code will be checked to be well typed now!
  method N m1(N that) that+1N
  method N m2(N that, D other) other.m1(that)+1N
  }   
\end{lstlisting}
\end{NiceCode}
\end{frame}


\begin{frame}[fragile]
Is this code still ok? why or why not?
\begin{NiceCode}{59ex}{0.8}
\begin{lstlisting}
C:{    'this code is required to be completely well typed 
         'before D can be generated
  type method Library a() {
    method N m1(N that) that+1N
    }
  }
D:Compose[ C.a() ] << {'code not required to be checked at this stage.
  method N m2(N that, D other) this.m1(that)+1N
  } 'note, no declaration for m1!
\end{lstlisting}
\end{NiceCode}
rewrites to 

\begin{NiceCode}{59ex}{0.8}
\begin{lstlisting}
C:{
  type method Library a() {
    method N m1(N that) that+1N
    }
  }
D:{'this code is now available to type-check
  method N m1(N that) that+1N
  method N m2(N that, D other) this.m1(that)+1N
  }   
\end{lstlisting}
\end{NiceCode}
\end{frame}


\begin{frame}[fragile]
What happens here?
\begin{NiceCode}{59ex}{0.8}
\begin{lstlisting}
D:Compose[{  method T foo()...}]<<
  Refactor::RenameSelector[ Selector"foo()" to: Selector"bar()" ] <<{
    method D foo() this
    method Any m1() this.foo().foo()
    }
\end{lstlisting}
\end{NiceCode}
rewrites to 
\begin{NiceCode}{59ex}{0.8}
\begin{lstlisting}
D:Compose[{  method T foo()...}]<< {
    method D bar() this
    method Any m1() this.bar().foo()
    }
\end{lstlisting}
\end{NiceCode}
rewrites to 
\begin{NiceCode}{59ex}{0.8}
\begin{lstlisting}
D:{
  method T foo() ...
  method D bar() this
  method Any m1() this.bar().foo()
  }
\end{lstlisting}
\end{NiceCode}
Here \Q@this.foo()@ returns \Q@D@, and at the start we do not know what methods
\Q@D@ will have.
\end{frame}



\begin{frame}[fragile]
Sometimes "typish" information is needed for the operators to work
\begin{NiceCode}{59ex}{0.8}
\begin{lstlisting}
D:Refactor::RenameSelector[ Selector"m()" to: Selector"k()" ]<< {
  method Outer0 m() this.m()
  method D toD() this
  method Outer0 m0() Foo().m()
  method Outer0 m1() this.m().m()
  method Outer0 m2() this.toD().m()
  Nested:{  type method Outer1 id(Outer1 that) that
            type method D toD(Outer1 that) that  }
  method Outer0 m3() Nested.id(this).m()
  method Outer0 m4() Nested.toD(this).m()
  }
\end{lstlisting}
\end{NiceCode}
rewrites to 

\begin{NiceCode}{59ex}{0.8}
\begin{lstlisting}
D:{
  method Outer0 k() this.k()'renamed 2 times
  method D toD() this
  method Outer0 m0() Foo().m()
  method Outer0 m1() this.k().k()'renamed 2 times
  method Outer0 m2() this.toD().m()
  Nested:{  type method Outer1 id(Outer1 that) that 
            type method D toD(Outer1 that)that     }
  method Outer0 m3() Nested.id(this).k()'renamed
  method Outer0 m4() Nested.toD(this).m()
  }
\end{lstlisting}
\end{NiceCode}
\end{frame}


\begin{frame}[fragile]
\frametitle{Final example}
What happens here?
\begin{NiceCode}{59ex}{0.8}
\begin{lstlisting}
D:Refactor::RenameSelector[ Selector"m()" to: Selector"k()" ] << {
  method Outer0 m1() this.m().m()
  }
\end{lstlisting}
\end{NiceCode}

If \Q@this.m()@ returns \Q@Outer0@ I have to rename both calls,
otherwise only the first one.
But there is no \Q@method Outer0 m()@
in the code.

Trying to rename a method that is not declared:
we decided to treat this as an exception.

\end{frame}



