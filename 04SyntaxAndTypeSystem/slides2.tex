
\begin{frame}[fragile]
\frametitle{Detailed Language introduction}
\begin{itemize}
\item Numbers, String and unit of measures
\item Collections and With
\item Mutually recursive bindings
\item Exceptions/return
\item Control flow
\item Support for multiple files
\item type modifiers
\item modifiers immutable and mutable
\item modifiers lent and readable
\item modifier capsule
\item modifier type
\end{itemize}
\end{frame}


\begin{frame}[fragile]
\frametitle{Numbers, String and unit of measures}
Numbers and strings are explicitly typed, allowing a higher level of checking.
\\
We support unit of measure and string formatting.
 For example:
\begin{NiceCode}{59ex}{0.8}
\begin{lstlisting}
a=12Meter + 4Meter $\Comment{that is fine}$
b=12Meter + 4Kg $\Comment{type error}$
c=12Meter * 4Kg $\Comment{type error too}$
d=KgMeter[12Meter; 4Kg] $\Comment{that is fine}$
e=S"just a string"
f=URL"www.aStringRespectingSomeRules"
g=Prolog"
  'isNum(z). %my multiline prolog string
  'isNum(s(X)):-isNum(X).
  "
\end{lstlisting}
\end{NiceCode}
Pure object oriented model: we do not have any primitive type, nor predefined classes.
All the operators are desugared as method calls,  e.g. \Q@12Meter + 4Meter@ $\rightarrow$ 
\Q@Meter.#parseNumber(...).#plus(Meter.#parseNumber(...))@
\end{frame}


\begin{frame}[fragile]
\frametitle{Collections Initialization}
Collection classes support the \Q@[...;...;]@ initialization syntax
\begin{NiceCode}{59ex}{0.8}
\begin{lstlisting}
v=NVector[1N;2N;3N;] $\Comment{a vector of N (natural numbers)}$
\end{lstlisting}
\end{NiceCode}
A matrix can be initialized in a similar way
\begin{NiceCode}{59ex}{0.8}
\begin{lstlisting}
m=NMatrix2d[
     #[1N; 2N; 3N; 4N ];
     #[1N; 2N; 3N; 4N ];
     #[1N; 2N; 3N; 4N ];
     ] 
\end{lstlisting}
\end{NiceCode}
Maps can be initialized in the following ways:
%'last ; is allowed but not required
\begin{NiceCode}{59ex}{0.8}
\begin{lstlisting}
myMap1=SNMap[ $\Comment{provide key and val for all the entries}$
  key: S"Mike" val: 12N; $\Comment{S is the string type}$
  key: S"Luke" val: 10N; $\Comment{key and val are just parameters names}$
  ]
$\Comment{or equivalently, using :: to access nested classes}$
entry=SNMap::Entry $\Comment{entry refers to the class SNMap::Entry}$
myMap2=SNMap[$\Comment{entry objects has key and val fields}$
  entry(key: S"Mike" val: 12N);  entry(key: S"Luke" val: 10N);]
\end{lstlisting}
\end{NiceCode}
\end{frame}



\begin{frame}[fragile]
\frametitle{\Q@with@: a Swiss army knife to encode complex behaviour}
There are two basic usage: as for-each and as a typecase.
\begin{NiceCode}{59ex}{0.8}
\begin{lstlisting}
vec= SVector[S"foo"; S"bar"; S"beer"]
var S result=S""
with myElem in vec.vals() (result:=result++myElem)$\Comment{like for(myElem:vec)\{..\}}$
$\Comment{result==S"foobarbeer"}$
with myData=foo.bar() ($\Comment{like a typecase/switch/chain of instanceof}$
  on S  ( Gui.alert(myData) )  $\Comment{if is a string}$
  on N case 2N.divides(myData) ( Gui.alert(...) ) $\Comment{if is an even natural}$
  )
\end{lstlisting}
\end{NiceCode}
if \Q@myData@ is already declared one can simply write
\begin{NiceCode}{59ex}{0.8}
\begin{lstlisting}
with myData (
  on S ( Gui.alert(myData) )  
  default (error S"myData was not a string") 
  )
\end{lstlisting}
\end{NiceCode}
Those two modes can be combined
\begin{NiceCode}{59ex}{0.8}
\begin{lstlisting}
vec= AnyVector[S"foo"; 12N; S"beer";]
var S result=S""
with myElem in vec.vals() (on S ( result:=result++myElem ) ) 
$\Comment{result==S"foobeer", composed by all strings inside vec}$
\end{lstlisting}
\end{NiceCode}
\end{frame}


\begin{frame}[fragile]
\frametitle{\Q@with@: a Swiss army knife to encode complex behaviour}
\Q@with@ can be used as list comprehension
\begin{NiceCode}{59ex}{0.8}
\begin{lstlisting}
vec=AnyVector[S"foo"; 12N; S"beer";]
v=SVector[with myElem in vec.vals() (on S myElem )]$\Comment{filter non strings}$
$\Comment{v==SVector[S"foo"; S"beer";]}$
\end{lstlisting}
\end{NiceCode}
for multiple dispatch:
\begin{NiceCode}{59ex}{0.8}
\begin{lstlisting}
method N m(Shape x, Person y, Vehicle z) { $\Comment{example of method using with}$
 with x y z (
 on Square Student Car (...return ...)$\Comment{x here is a Square}$
 on Circle Person Airplane (...)$\Comment{x here is a Circle}$
 default (...)$\Comment{default case, here x is just a Shape}$ 
  )}
\end{lstlisting}
\end{NiceCode}
Or to iterate over multiple collections at once
\begin{NiceCode}{59ex}{0.8}
\begin{lstlisting}
rs=Vector[1N;2N;3N;]
as= Vector[10N;20N;30N;]
bs= Vector[100N;200N;300N;]
$\Comment{here a, b and r iterate over my data}$
with a in as.vals(), b in bs.vals(), var r in rs.vals() (r:=r+a+b)
$\Comment{now rs==Vector[111N;222N;333N;]}$
\end{lstlisting}
\end{NiceCode}
\end{frame}

\begin{frame}[fragile]
\frametitle{The \Q@with@: a Swiss knife to encode complex behaviour}
While iterating on multiple collections, a dynamic error is usually raised if \Q@rs@, \Q@as@ and \Q@bs@ have different length.
This behaviour can be tuned in many way, for example using \Q@cut()@ and \Q@fill(...)@

\begin{NiceCode}{59ex}{0.8}
\begin{lstlisting}
rs=Vector[1N;2N;3N;]
as= Vector[10N;20N;30N;40N;50N;]
bs= Vector[100N;200N;]
with a in as.valsCut(), b in bs.vals(fill:300N), var r in rs.vals() (
  r:=r+a+b)
$\Comment{rs==Vector[111N;222N;333N;]}$
\end{lstlisting}
\end{NiceCode}
Resources used within an iteration can be released after the iteration 
since collections are notified when the iteration ends.
\begin{NiceCode}{59ex}{0.8}
\begin{lstlisting}
$\Comment{a contains "foo1 \textbackslash n foo2 \textbackslash n foo3"}$
$\Comment{b contains "bar1\textbackslash n bar2"}$
with 
  input in LineStream.readFile(S"a"), 
  var output in LineStream.readWriteFile(S"b",fill:S"None") (
    output:= output +" : "+input)$\Comment{line by line, add input in the file}$
$\Comment{b contains "bar1 : foo1 \textbackslash n bar2 : foo2 \textbackslash n None : foo3"}$
\end{lstlisting}
\end{NiceCode}
\end{frame}



\begin{frame}[fragile]
\frametitle{Mutually recursive local bindings}
Like OCaML ``let rec'', Haskell lazy bindings or Placeholers

\begin{NiceCode}{59ex}{0.8}
\begin{lstlisting}
{
A:{(B b)}
B:{(A a)}
...
  ($\Comment{a parentesized set of expressions is a `block'}$
    A myA=A(b:myB)$\Comment{using the binding b that is declared "at the same time"}$
    myB=B(a:myA) $\Comment{note that the local binding type can always be inferred}$
    myA.b().a().b() $\Comment{this is the result of the block}$
  )
}
\end{lstlisting}
\end{NiceCode}

Variables declared in the same scope can see each other.
The type system ensures that variables are never used as receivers until they are initialized.
It also prevents cases like \Q@T x=x@ and all the subtle variants of it.
\\*
(We will not see this part of the type system today.)
%Variables declared with \Q@var@ can be updated.
\end{frame}



\begin{frame}[fragile]
\frametitle{Single dispatch subtype polymorphism}
\begin{NiceCode}{59ex}{0.8}
\begin{lstlisting}
PredOnZero: {()}
Nat: {interface method Nat pred();}
Zero:{()<:Nat method pred() {error PredOnZero()} }
Succ: {(Nat $\_$pred)<:Nat  method pred() this.$\_$pred()}
\end{lstlisting}
\end{NiceCode}
Note that there is no overloading: the added flexibility on method selectors provides
a better solution to the need of reusing names:
\begin{itemize}
\item \Q@A.foo()@, \Q@A.foo(bar:x)@, \Q@A.foo(beer:x)@
are all different methods, with different method selectors.
\item \Q@A.#foo()@ is a valid, different method selector too.
Conventionally, \Q@#@  denote \emph{warning}, on methods to use only while knowing what you are doing, like methods opening resources that needs to be manually closed or methods exposing internal mutable state.
\end{itemize}
All the operators and control flow constructs of the language can be expressed as combination of 
method calls and (near conventional) exceptions.
\end{frame}

\begin{frame}[fragile]
%\addtocounter{framenumber}{-1}
\frametitle{Multi-sort result 
propagation/inspection}
Like Exceptions in Java

\begin{NiceCode}{59ex}{0.8}
\begin{lstlisting}

PredOnZero: {()}
Nat: {interface method Nat pred();}
Zero:{()<:Nat method pred() {error PredOnZero()} }
Succ: {(Nat $\_$pred)<:Nat  method pred() this.$\_$pred()}
..
  method foo(Nat n) {
    Nat n1=n.pred() $\Comment{Nat is the preferred result of pred()}$
    catch error x ($\Comment{catch - on is similar to a try-catch}$
      on PredOnZero (...)$\Comment{PredOnZero is an error result}$
      )
    ..$\Comment{here you can use n1}$
    }
\end{lstlisting}
\end{NiceCode}

{\small
\Q@exception@s are like checked exceptions in Java;
\Q@error@s are like Java unchecked ones, but capturing them does not expose half processed objects;
\Q@return@s are guaranteed to be captured in the scope of their method, and serve the role of Java return keyword.
%Since we have exceptions, we only need a \Q@loop (...)@ operation in order to be able to encode all desired control structure.
}
\end{frame}

\begin{frame}[fragile]
\frametitle{Exceptions}
In real systems every operation can fail (as method call can cause
 a in stack-overflow no matter what).
To support critical applications in real systems
 we need the ability to capture such (any) errors and resume the application.
 Only exceptions seem to be able to provide such a capability.
\\*${}_{}$\\*
In 42 exceptions are related to variable declarations:
\begin{itemize}
\item No explicit \alert{try}.
\item Failure to initialize a variable is an exception.
\item The variables can be used if no exception is thrown.
\item Body of selected \Q@catch on@ do not see uninitialized variables.
\item \Q@exception@ correspond to Java checked exceptions.
\item \Q@error@ preserve strong exception safety.
\item \Q@return@ can not escape body of a method.
\end{itemize}
\end{frame}

\begin{frame}[fragile]
\frametitle{Return}
Expressions enclosed in \Q@{ .. }@ can use \Q@return@
\begin{NiceCode}{59ex}{0.8}
\begin{lstlisting}
method S getName(){
  if condition ( return S"No Name" )
  S userName={if !User.registered() (return S"")
    return User.getName()
    }  
  return S"The name is "++userName
  }
\end{lstlisting}
\end{NiceCode}
Nested curly brackets can be used to obtain nested return scopes.
This reduces the needs for continue, and break is obtained by
\Q@exception void@

\begin{NiceCode}{59ex}{0.8}
\begin{lstlisting}
method S getName(){
  while condition {
    if skipCase (return void)
    if searchComplete (exception void)
    doStuff
    }
\end{lstlisting}
\end{NiceCode}

\end{frame}

\begin{frame}[fragile]
\frametitle{Control flow}
All is control flow is desugared as combinations of 
method calls and exceptions.
For example: \Q@if@
\\*
\begin{NiceCode}{26ex}{0.8}
\begin{lstlisting}
if x (doStuff1)      $\longrightarrow$
else (doStuff2)      $\longrightarrow$
                     $\longrightarrow$
${}_{}$
\end{lstlisting}
\end{NiceCode}
\begin{NiceCode}{26ex}{0.8}
\begin{lstlisting}
(x.checkTrue()
 catch exception(
    on Void (doStuff2))
doStuff1)  
\end{lstlisting}
\end{NiceCode}

\begin{NiceCode}{26ex}{0.8}
\begin{lstlisting}
if e (doStuff1)      $\longrightarrow$
${}_{}$
\end{lstlisting}
\end{NiceCode}
\begin{NiceCode}{26ex}{0.8}
\begin{lstlisting}
if e (doStuff1)
else (void)
\end{lstlisting}
\end{NiceCode}

\begin{NiceCode}{26ex}{0.8}
\begin{lstlisting}
if e (doStuff1)      $\longrightarrow$
else (doStuff2)      $\longrightarrow$
\end{lstlisting}
\end{NiceCode}
\begin{NiceCode}{26ex}{0.8}
\begin{lstlisting}
(x=e if x (doStuff1)
else (void))
\end{lstlisting}
\end{NiceCode}

\Q@{ .. return .. }@

\begin{NiceCode}{26ex}{0.8}
\begin{lstlisting}
T x={ .. }      $\longrightarrow$
                $\longrightarrow$
${}_{}$
\end{lstlisting}
\end{NiceCode}
\begin{NiceCode}{26ex}{0.8}
\begin{lstlisting}
T x=(T y={ .. } 
  catch return z (on T z)
  y)
\end{lstlisting}
\end{NiceCode}
\end{frame}

\begin{frame}[fragile]
\frametitle{Outers}
In 42 all class names denote a node in relation 
to where they are defined.
Names starting with an \Q@Outer@ are explicit;
other names have a sensible \Q@Outer@ level inferred at pre-compilation phase. That is

\begin{NiceCode}{59ex}{0.8}
\begin{lstlisting}
{
A: {
  A: {(B b)
    B: {(C c)}
    C: {(A a, Outer3::A a3)}
    }
  }}
\end{lstlisting}
\end{NiceCode}

is equivalent to 

\begin{NiceCode}{59ex}{0.8}
\begin{lstlisting}
{
A: {
  A: {(Outer0::B b)
    B: {(Outer1::C c)}
    C: {(Outer2::A a, Outer3::A a3)}
    }
  }}
\end{lstlisting}
\end{NiceCode}
\end{frame}



\begin{frame}[fragile]
%\addtocounter{framenumber}{-1}
\frametitle{Support for multiple files}
Little 42 programs can consist of a single file with \texttt{.L42}
extension.\\*
Larger programs are a folder with subfolders.
Every folder must include a file \texttt{Outer0.L42}
containing the program code.
%The special expression
%``\Q@...@'' refers to other files/folders in the current folder.
For example in 
file MyGame/Outer0.L42
\begin{NiceCode}{59ex}{0.8}
\begin{lstlisting}
reuse $\ReuseUrl{L42.is/fun}$
Cell:Enumeration"Rock Grass Exit"
MapCell: Collections.matrix(Cell)
MyMap: ... $\Comment{the symbol ... means search for a file/dir called MyMap}$
Gui.alert(MyMap().toText())
\end{lstlisting}
\end{NiceCode}

file MyGame/MyMap.L42 or file MyGame/MyMap/Outer0.L42
\begin{NiceCode}{59ex}{0.8}
\begin{lstlisting}
{ type method(){
  r=Cell::Rock
  g=Cell::Grass
  e=Cell::Exit
  return MapCell[ #[r;r;r;r];$\Comment{MapCell is a matrix of Cells}$
    #[r;g;g;r];
    ...
    #[r;r;e;r]]
}}
\end{lstlisting}
\end{NiceCode}
\end{frame}

%\begin{frame}[fragile]
%\addtocounter{framenumber}{-1}
%\frametitle{Resource}
%
%Using \Q@Resource@ is like to declare a class with a single method,
% returning such resource value.
%Resources are much safer then static fields.
%\end{frame}
%


%\fullPageImage{lego1}

\begin{frame}[fragile]
\frametitle{Type modifiers and promotion}
Up to now we mostly explored the functional part of 42.
\\*
We also offer conventional mutable state (as in \Q@Person:{mut (var Year age) ...}@ )
\\*
However, since we expect libraries to be automatically composed, we offer \alert{type modifiers} as an instrument to prevent multiple libraries to corrupt each other data-structures.
\\*
For each class (\Q@Person@ as in the example) there are multiple types, encoding the different access permissions.
\end{frame}


\begin{frame}[fragile]
\frametitle{Type modifiers and promotion}
\begin{Scaled}{0.9}{0.9}
\!\!\!\!
\parbox{67ex}{
%There are three kinds of objects in 42:
%(deeply) immutable objects as found in many functional languages,
%mutable objects as in Java and type objects (representing classes and interfaces).
%Fields, updateable fields, bindings and variables are called references and
%can refer to each kind of objects using the following type modifiers:
\begin{itemize}
\item \Q@Person p@ is an immutable reference, and refer to a deeply immutable object graph (as in many functional languages).

\item \Q@mut Person p@ is a mutable reference, and refer to an unrestricted mutable object (the Java default).

\item \Q@lent Person p@ is an hygienic reference to mutable objects: can be used 
for mutation but it can not be stored in existing structures for long term use,nor
 can pre-existing mutable objects be referenced by it.

\item \Q@read Person p@ is a readable reference, and is a common supertype for both mutable, lent and immutable references.
A readable reference can not be use for mutation nor stored inside existing structures for long term reuse.

\item \Q@capsule Person p@ is a fully encapsulated reference: it refers to
 a completely isolated object graph, reachable only through that reference.
 Thus \Q@capsule@ references generate the object as mutable or
 immutable depending on the way the capsule is opened.

\item\Q@type Person p@ Finally, the type references refers to the Class itself.
\end{itemize}
}
\end{Scaled}
\end{frame}

\begin{frame}[fragile]
\frametitle{Immutable objects}
Deeply immutable,
no matter how the fields are declared, always immutable values are fetched out of fields.
Note the following code, mixing immutable and mutable content
\begin{lstlisting}
Pathologies:Collections.list(of:Pathology)
Person:{mut (var Date lastVisit, mut Pathologies history)}
Main:{
  me=Person(lastVisit:Date(....),
      history:Pathologies[Laziness()])
  me.lastVisit(Date.today()) $\Comment{ok replace the whole Date}$
$\Comment{me.lastVisit().setDay(1Day) 'wrong change a part of Date}$
  me.history().add(DustAllergy())$\Comment{ok change a part of history}$
$\Comment{me.history(Pathologies[]) 'wrong change the whole history}$
  ....}
  \end{lstlisting}
  
\Q@me.lastVisit(Date.today())@ is valid code; the \Q@Date@ object is immutable; but is  referenced
 with a variable field (\Q@var lastVisit@ in the example), that can be mutated. The opposite holds for the medical history.
\end{frame}

\begin{frame}[fragile]
\frametitle{Mutable references}
Mutable references refer to unrestricted mutable objects like in Java.
Newly created objects have the modifier declared in the header, but a context sensitive promotion allows to convert
mutable references to capsule or immutable.
\\*
\begin{NiceCode}{59ex}{0.8}
\begin{lstlisting}
{
Person:{mut (var mut Person friend)
  method Person factory(){ $\Comment{return type is (immutable) Person}$
    mut Person fred=Person(friend:fred)
    mut Person barney=Person(friend:barney)
    fred.friend(barney)  
    barney.friend(fred)
    return fred $\Comment{mut Person promoted to (immutable) Person}$
   $\Comment{now fred and barney are friends forever!}$
    }
  }
}
\end{lstlisting}
\end{NiceCode}

\end{frame}



\begin{frame}[fragile]
\frametitle{Lent and Readable references}
Hygienic reference to mutable objects: can be used 
for mutation but it can not be stored in existing structures for long term use,nor pre-existing mutable objects can be referenced by it. Useful to preserve encapsulation properties of objects.

\begin{NiceCode}{59ex}{0.8}
\begin{lstlisting}
lent C treasure=myLovedData.getStuff()
lent C r=evilCode.useForAWhile(treasure)
$\Comment{now my treasure is back!}$
$\Comment{note, r could still refer to treasure}$
$\Comment{but evilCode can not.}$
\end{lstlisting}
\end{NiceCode}
A method taking a lent reference must release it after completion.
Providing conceptually private property objects as lent preserves encapsulation.
Bindings and variables can be lent;
mutable objects with lent fields are discussed next.


Readable references simply combine the restrictions of Lent and Immutable. They closely model the idea of a safe ``parameter''.

\end{frame}

\begin{frame}[fragile]
%\addtocounter{framenumber}{-1}
\frametitle{Temporary objects}
Temporary objects, i.e., objects that are created to compute a result but
are meant to be garbage-collected after the computation,
are best referenced as \Q@lent@ objects. 

For any expression, internally created objects used for the computation of the result but garbage collected when the expression evaluation is completed are called temporary objects; and the \Q@lent@ modifier helps to track of them!

Mutable objects with lent fields are always referred by  \Q@lent@ references, and are
an useful type of 
temporary objects; 
useful for collections of lent objects coming from different sources
and for the decorator pattern.

\begin{NiceCode}{59ex}{0.8}
\begin{lstlisting}
lent C a=...   lent C b=...   lent C c=...
lent LentVec tempObj=LentVec[a;b;c;]$\Comment{temporary collection}$

var lent C a=...
a:=Decorator.addLogging(a) $\Comment{temporary decorator}$
\end{lstlisting}
\end{NiceCode}

\end{frame}

\begin{frame}[fragile]
\frametitle{Capsules}
We support encapsulation, indeed we have the \Q@capsule@ modifier!
A \Q@capsule@ can not be accessed at all, thus is not observable
 whether the content is mutable or immutable.

The only way to obtain a \Q@capsule@ is converting a  \Q@mut@ reference by promotions.
A \Q@capsule@ reference can then be transparently converted into other references.
The next example shows how a \Q@capsule@ can be converted into a \Q@mut@, mutated and then converted back to \Q@capsule@.
\begin{NiceCode}{59ex}{0.8}
\begin{lstlisting}
method
capsule C playWithCapsules(capsule C x, C p1, lent C p2, read C p3){
    mut C myX=x
    $\Comment{here you can do all you want with the stuff in scope}$
    return myX
    }
\end{lstlisting}
\end{NiceCode}
Capsules local binding can be used only once,
How to support capsule fields in an open research question.
%but class instrumenters can impose  opportune relaxations of
%the capsule guarantee for some \Q@mut@ fields.
\end{frame}

\begin{frame}[fragile]
\frametitle{Promotions as a form of alchemy }
Since 42 do not have any form of static variables,\\*
any expression that produce a mutable reference (the bad dangerous and low value stuff)\\*
but do not takes in input any mutable reference\\*
must create such mutable value during its own execution.\\*
Thus, we can convert such mutable reference to \Q@capsule@ (the gold!)

\end{frame}

\begin{frame}[fragile]
\frametitle{Type: Instance, class and metaclass}
Classes have instances, but classes are not types.
\\*
Type=type modifier + class name
\\*${}_{}$\\*
For example mutable and immutable Persons
have different types,
so if I declare \Q@mut Person p1=..@ or \Q@Person p2=..@,
in both cases I get an instance of \Q@Person@,
 but with different types:
there is a radically different set of methods that I can call
on \Q@p1@ w.r.t. \Q@p2@.
In the same way in \Q@type Person p0=Person@
\Q@p0@ is still an instance of \Q@Person@.
\\*${}_{}$\\*
\Q@Person@ is a special instance that ``just exists''
and is visible everywhere, or is recreated everywhere is needed, if you prefer.
\Q@p0@ offers a radically different set of methods w.r.t. \Q@p1@ or \Q@p2@,
but that is fine since it has a different type.\\*
So the class of \Q@p1@ is \Q@Person@, and the class of \Q@Person@ is \Q@Person@.
\\*${}_{}$\\*
Metaclasses are supported as library following the mirrors philosophy.
\end{frame}


\begin{frame}[fragile]
\frametitle{Type and subyping}
With an interface \Q@I@,
\Q@type I@ would reference only type-objects implementing \Q@I@.
For example:
\begin{NiceCode}{59ex}{0.8}
\begin{lstlisting}
{
I:{interface $\Comment{type methods accept type receivers; feels like static in Java}$
  type method I (I^ f); $\Comment{ \^{} means placeholder, a not yet existing object}$
  method I f(); $\Comment{getter}$
  }
A:{new(I f)<:I, method(f){return this.new(f:f)}}$\Comment{forwarding constructors }$
B:{new(I f)<:I, method(f){return this.new(f:f)}}$\Comment{to the method of I}$
User:{()
  method I init(type I t1, type I t2) (
    i1=t1(f:i2)
    i2=t2(f:i1)
    i1
    )
  method I user() (this.init(t1:A t2:B)) $\Comment{here A and B are the only two}$
           $\Comment{values acceptable for a type I parameter}$
    }}
\end{lstlisting}
\end{NiceCode}
\end{frame}

%\begin{frame}[fragile]
%\frametitle{Owned fields}
%An owned field can be initialized and updated only using a \Q@capsule@.
%Intuitively an owned field contains a ``balloon'' of data, that is reachable only through that field itself.
%However, some of that data could be currently lent, or ``already accessed using that field''.
%Precisely, ignoring lent references and the owned field itself,
% objects connected to the owner are called external, while object connected to the root object of the balloon are called internal, and those two sets are disjoint.
%
%The main use for owned field is to guarantee to an object the control of its fields. We expect most fields of shared object to be either immutable or owned.
%\begin{NiceCode}{59ex}{0.8}
%\begin{lstlisting}
%C:{(owned D f)}
%...
%C(D())'ok, since D() can be promoted to capsule in place
%...
%method (capsule D x)(...
%  C(x)'ok too!
%  )
%\end{lstlisting}
%\end{NiceCode}
%\end{frame}


\begin{frame}[fragile]
\frametitle{That is all for now. What is \textbf{not} here?}
What we are proud to not have
\begin{itemize}
\item primitive types
\item primitive classes (only primitive nodes are Library, Void and Any)
\item predefined literals (only literal for code values)
\item static fields and related initialization issues (circularity + exception)
\item null and default values in general
\item concept of top level class / absolute names
\item extends / base class
\item variable / field hiding
\item closures (especially, no capture of variable)
\item super / super-constructors
\item inner classes (only nested)
\end{itemize}

\end{frame}


\begin{frame}[fragile]
\frametitle{That is all for today. What is \textbf{not} here?}

%What I had no time to explain today
\begin{itemize}
%\item Exceptions
%\item placeholder types for circular initialization
\item sand-boxing  (integrated with the language, fundamental for security)
\item connection with the rest of the world (as native interface in Java)
\item code and IDE connection: libraries can personalize the IDE
\item how the syntactic sugar works
%\item 42 project management
\item fully automatic parallelization
\item full abstract interpretation/symbolic execution as optimization
\end{itemize}

\end{frame}

\begin{frame}[fragile]
\frametitle{ Thanks! }
\center{\huge Questions?}
\end{frame}

\begin{frame}[fragile]
\addtocounter{framenumber}{-1}
\frametitle{That is all. What is \textbf{not} here?}

What can be added by libraries written in 42 itself
\begin{itemize}
\item booleans
\item numbers
\item arrays
\item multi-methods
\item access to libraries written in other languages
\item collections/generics
\item enumerations
\item pre-post conditions verification
\item deployment on multiple platforms
\item optimization(automatic parallelism, method inlining)
\item ...
\end{itemize}
\end{frame}




%
\begin{frame}[fragile]
\addtocounter{framenumber}{-1}
\frametitle{The syntactic sugar}
\begin{itemize}
\item reuse is (conceptually) replaced with the library code itself
\item $\Comment{@private}$ members are turned in unforgeable names during reuse replacement.
\item typed literals (\Q@12Meter@, \Q@DBConnection"...."@,..)
are desugared in sequences of method calls, something like \Q@Meters.#parseInt().#$\_$1().#$\_$2().#endParse()@
\item if, while, with, \{ \} and so on are defined by syntactic sugar.
\end{itemize}


\end{frame}



\begin{frame}[fragile]
\addtocounter{framenumber}{-1}
\frametitle{Target usage}
A small set of good programmers can design good libraries that can be used by a mass of
non-professional programmers to obtain decent but non cutting-edge applications.
Good application designers 
(user interactions? game designer? data presentation experts?)
will be able to use 42 without great programming skills.

Good programmers will find it easier to develop libraries in 42 than in other languages.
\end{frame}


\begin{frame}[fragile]
\addtocounter{framenumber}{-1}
\frametitle{Deep instantiation example}
Collections generation can be based on desired properties.
\begin{NiceCode}{59ex}{0.8}
\begin{lstlisting}
{...
S:{...}$\Comment{S is a string type}$
Collection:{...}$\Comment{Collection(dept4) generate different collection kinds}$
Set:Collection($\Comment{Set(dept3) generate sets for different contents}$
  order:Collection::Order::Irrelevant
  repetition:Collection::Repetition::SilentlyAvoided
  size:Collection::Size::Dynamic
  ) $\Comment{here Set.of(....) is a kind of collection where....}$
NameSet:Set.of(S)$\Comment{NameSet(dept2) is a set of immutable strings}$
$\Comment{Set.ofShared(....) will create sets of shared objects}$
SetKindSet:Set.ofType(Set::Kind)
$\Comment{SetKindSet(dept2) set of sets type objects}$
CollectionKindSet:Set.ofType(Collection::Kind)
$\Comment{CollectionKindSet(dept2) set of Collections type objects}$
myFriends:NameSet[S"Mattia";S"Luca";S"Paolo"]
$\Comment{myFriends(dept1) is a specific set of names}$
mySets:SetKindSet[NameSet;SetKindSet]
$\Comment{those are all the kind of sets I have defined}$
myGenerators:CollectionKindSet[Set]
$\Comment{Here I defined 1 kind of collections only: Set}$
}
\end{lstlisting}
\end{NiceCode}
\end{frame}



\begin{frame}[fragile]
\addtocounter{framenumber}{-1}
\frametitle{5 pillars / Core Language mechanisms}
\begin{itemize}
\tiny
\item Active libraries
\begin{itemize}
\tiny
\item library literals and code composition algebra;
\item incremental typing;
\item meta level soundness.
\end{itemize}

\item Minimal OO capabilities:
\begin{itemize}
\tiny
\item (final) classes with state (constructor composed by fields) and behaviour;
\item single dispatch method call, with nominal method selectors;
\item interfaces as the only way to induce subtyping; that is, all class types are exact types.
\end{itemize}
\item Signals:
\begin{itemize}
\tiny
\item three kinds of alternative results (error, exception and return), plus conventional (preferred) results;
\item capturing of alternative results based on the object type;
\item safe interaction between mutable state and signals.
\end{itemize}

\item Multiple variable declarations:
\begin{itemize}
\tiny
\item placeholder semantics;
\item placeholder types;
\item commitment points.
\end{itemize}

\item Encapsulation by (deep) type modifiers:
\begin{itemize}
\tiny
\item object kinds: immutable, mutable and type;
\item references: immutable, shared, readable, lent, capsule and owned;
\item transparent promotions.
\end{itemize}

\end{itemize}
\end{frame}



